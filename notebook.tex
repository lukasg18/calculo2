
% Default to the notebook output style

    


% Inherit from the specified cell style.




    
\documentclass[11pt]{article}

    
    
    \usepackage[T1]{fontenc}
    % Nicer default font (+ math font) than Computer Modern for most use cases
    \usepackage{mathpazo}

    % Basic figure setup, for now with no caption control since it's done
    % automatically by Pandoc (which extracts ![](path) syntax from Markdown).
    \usepackage{graphicx}
    % We will generate all images so they have a width \maxwidth. This means
    % that they will get their normal width if they fit onto the page, but
    % are scaled down if they would overflow the margins.
    \makeatletter
    \def\maxwidth{\ifdim\Gin@nat@width>\linewidth\linewidth
    \else\Gin@nat@width\fi}
    \makeatother
    \let\Oldincludegraphics\includegraphics
    % Set max figure width to be 80% of text width, for now hardcoded.
    \renewcommand{\includegraphics}[1]{\Oldincludegraphics[width=.8\maxwidth]{#1}}
    % Ensure that by default, figures have no caption (until we provide a
    % proper Figure object with a Caption API and a way to capture that
    % in the conversion process - todo).
    \usepackage{caption}
    \DeclareCaptionLabelFormat{nolabel}{}
    \captionsetup{labelformat=nolabel}

    \usepackage{adjustbox} % Used to constrain images to a maximum size 
    \usepackage{xcolor} % Allow colors to be defined
    \usepackage{enumerate} % Needed for markdown enumerations to work
    \usepackage{geometry} % Used to adjust the document margins
    \usepackage{amsmath} % Equations
    \usepackage{amssymb} % Equations
    \usepackage{textcomp} % defines textquotesingle
    % Hack from http://tex.stackexchange.com/a/47451/13684:
    \AtBeginDocument{%
        \def\PYZsq{\textquotesingle}% Upright quotes in Pygmentized code
    }
    \usepackage{upquote} % Upright quotes for verbatim code
    \usepackage{eurosym} % defines \euro
    \usepackage[mathletters]{ucs} % Extended unicode (utf-8) support
    \usepackage[utf8x]{inputenc} % Allow utf-8 characters in the tex document
    \usepackage{fancyvrb} % verbatim replacement that allows latex
    \usepackage{grffile} % extends the file name processing of package graphics 
                         % to support a larger range 
    % The hyperref package gives us a pdf with properly built
    % internal navigation ('pdf bookmarks' for the table of contents,
    % internal cross-reference links, web links for URLs, etc.)
    \usepackage{hyperref}
    \usepackage{longtable} % longtable support required by pandoc >1.10
    \usepackage{booktabs}  % table support for pandoc > 1.12.2
    \usepackage[inline]{enumitem} % IRkernel/repr support (it uses the enumerate* environment)
    \usepackage[normalem]{ulem} % ulem is needed to support strikethroughs (\sout)
                                % normalem makes italics be italics, not underlines
    

    
    
    % Colors for the hyperref package
    \definecolor{urlcolor}{rgb}{0,.145,.698}
    \definecolor{linkcolor}{rgb}{.71,0.21,0.01}
    \definecolor{citecolor}{rgb}{.12,.54,.11}

    % ANSI colors
    \definecolor{ansi-black}{HTML}{3E424D}
    \definecolor{ansi-black-intense}{HTML}{282C36}
    \definecolor{ansi-red}{HTML}{E75C58}
    \definecolor{ansi-red-intense}{HTML}{B22B31}
    \definecolor{ansi-green}{HTML}{00A250}
    \definecolor{ansi-green-intense}{HTML}{007427}
    \definecolor{ansi-yellow}{HTML}{DDB62B}
    \definecolor{ansi-yellow-intense}{HTML}{B27D12}
    \definecolor{ansi-blue}{HTML}{208FFB}
    \definecolor{ansi-blue-intense}{HTML}{0065CA}
    \definecolor{ansi-magenta}{HTML}{D160C4}
    \definecolor{ansi-magenta-intense}{HTML}{A03196}
    \definecolor{ansi-cyan}{HTML}{60C6C8}
    \definecolor{ansi-cyan-intense}{HTML}{258F8F}
    \definecolor{ansi-white}{HTML}{C5C1B4}
    \definecolor{ansi-white-intense}{HTML}{A1A6B2}

    % commands and environments needed by pandoc snippets
    % extracted from the output of `pandoc -s`
    \providecommand{\tightlist}{%
      \setlength{\itemsep}{0pt}\setlength{\parskip}{0pt}}
    \DefineVerbatimEnvironment{Highlighting}{Verbatim}{commandchars=\\\{\}}
    % Add ',fontsize=\small' for more characters per line
    \newenvironment{Shaded}{}{}
    \newcommand{\KeywordTok}[1]{\textcolor[rgb]{0.00,0.44,0.13}{\textbf{{#1}}}}
    \newcommand{\DataTypeTok}[1]{\textcolor[rgb]{0.56,0.13,0.00}{{#1}}}
    \newcommand{\DecValTok}[1]{\textcolor[rgb]{0.25,0.63,0.44}{{#1}}}
    \newcommand{\BaseNTok}[1]{\textcolor[rgb]{0.25,0.63,0.44}{{#1}}}
    \newcommand{\FloatTok}[1]{\textcolor[rgb]{0.25,0.63,0.44}{{#1}}}
    \newcommand{\CharTok}[1]{\textcolor[rgb]{0.25,0.44,0.63}{{#1}}}
    \newcommand{\StringTok}[1]{\textcolor[rgb]{0.25,0.44,0.63}{{#1}}}
    \newcommand{\CommentTok}[1]{\textcolor[rgb]{0.38,0.63,0.69}{\textit{{#1}}}}
    \newcommand{\OtherTok}[1]{\textcolor[rgb]{0.00,0.44,0.13}{{#1}}}
    \newcommand{\AlertTok}[1]{\textcolor[rgb]{1.00,0.00,0.00}{\textbf{{#1}}}}
    \newcommand{\FunctionTok}[1]{\textcolor[rgb]{0.02,0.16,0.49}{{#1}}}
    \newcommand{\RegionMarkerTok}[1]{{#1}}
    \newcommand{\ErrorTok}[1]{\textcolor[rgb]{1.00,0.00,0.00}{\textbf{{#1}}}}
    \newcommand{\NormalTok}[1]{{#1}}
    
    % Additional commands for more recent versions of Pandoc
    \newcommand{\ConstantTok}[1]{\textcolor[rgb]{0.53,0.00,0.00}{{#1}}}
    \newcommand{\SpecialCharTok}[1]{\textcolor[rgb]{0.25,0.44,0.63}{{#1}}}
    \newcommand{\VerbatimStringTok}[1]{\textcolor[rgb]{0.25,0.44,0.63}{{#1}}}
    \newcommand{\SpecialStringTok}[1]{\textcolor[rgb]{0.73,0.40,0.53}{{#1}}}
    \newcommand{\ImportTok}[1]{{#1}}
    \newcommand{\DocumentationTok}[1]{\textcolor[rgb]{0.73,0.13,0.13}{\textit{{#1}}}}
    \newcommand{\AnnotationTok}[1]{\textcolor[rgb]{0.38,0.63,0.69}{\textbf{\textit{{#1}}}}}
    \newcommand{\CommentVarTok}[1]{\textcolor[rgb]{0.38,0.63,0.69}{\textbf{\textit{{#1}}}}}
    \newcommand{\VariableTok}[1]{\textcolor[rgb]{0.10,0.09,0.49}{{#1}}}
    \newcommand{\ControlFlowTok}[1]{\textcolor[rgb]{0.00,0.44,0.13}{\textbf{{#1}}}}
    \newcommand{\OperatorTok}[1]{\textcolor[rgb]{0.40,0.40,0.40}{{#1}}}
    \newcommand{\BuiltInTok}[1]{{#1}}
    \newcommand{\ExtensionTok}[1]{{#1}}
    \newcommand{\PreprocessorTok}[1]{\textcolor[rgb]{0.74,0.48,0.00}{{#1}}}
    \newcommand{\AttributeTok}[1]{\textcolor[rgb]{0.49,0.56,0.16}{{#1}}}
    \newcommand{\InformationTok}[1]{\textcolor[rgb]{0.38,0.63,0.69}{\textbf{\textit{{#1}}}}}
    \newcommand{\WarningTok}[1]{\textcolor[rgb]{0.38,0.63,0.69}{\textbf{\textit{{#1}}}}}
    
    
    % Define a nice break command that doesn't care if a line doesn't already
    % exist.
    \def\br{\hspace*{\fill} \\* }
    % Math Jax compatability definitions
    \def\gt{>}
    \def\lt{<}
    % Document parameters
    \title{Trabalho de Calculo II}
    
    
    

    % Pygments definitions
    
\makeatletter
\def\PY@reset{\let\PY@it=\relax \let\PY@bf=\relax%
    \let\PY@ul=\relax \let\PY@tc=\relax%
    \let\PY@bc=\relax \let\PY@ff=\relax}
\def\PY@tok#1{\csname PY@tok@#1\endcsname}
\def\PY@toks#1+{\ifx\relax#1\empty\else%
    \PY@tok{#1}\expandafter\PY@toks\fi}
\def\PY@do#1{\PY@bc{\PY@tc{\PY@ul{%
    \PY@it{\PY@bf{\PY@ff{#1}}}}}}}
\def\PY#1#2{\PY@reset\PY@toks#1+\relax+\PY@do{#2}}

\expandafter\def\csname PY@tok@w\endcsname{\def\PY@tc##1{\textcolor[rgb]{0.73,0.73,0.73}{##1}}}
\expandafter\def\csname PY@tok@c\endcsname{\let\PY@it=\textit\def\PY@tc##1{\textcolor[rgb]{0.25,0.50,0.50}{##1}}}
\expandafter\def\csname PY@tok@cp\endcsname{\def\PY@tc##1{\textcolor[rgb]{0.74,0.48,0.00}{##1}}}
\expandafter\def\csname PY@tok@k\endcsname{\let\PY@bf=\textbf\def\PY@tc##1{\textcolor[rgb]{0.00,0.50,0.00}{##1}}}
\expandafter\def\csname PY@tok@kp\endcsname{\def\PY@tc##1{\textcolor[rgb]{0.00,0.50,0.00}{##1}}}
\expandafter\def\csname PY@tok@kt\endcsname{\def\PY@tc##1{\textcolor[rgb]{0.69,0.00,0.25}{##1}}}
\expandafter\def\csname PY@tok@o\endcsname{\def\PY@tc##1{\textcolor[rgb]{0.40,0.40,0.40}{##1}}}
\expandafter\def\csname PY@tok@ow\endcsname{\let\PY@bf=\textbf\def\PY@tc##1{\textcolor[rgb]{0.67,0.13,1.00}{##1}}}
\expandafter\def\csname PY@tok@nb\endcsname{\def\PY@tc##1{\textcolor[rgb]{0.00,0.50,0.00}{##1}}}
\expandafter\def\csname PY@tok@nf\endcsname{\def\PY@tc##1{\textcolor[rgb]{0.00,0.00,1.00}{##1}}}
\expandafter\def\csname PY@tok@nc\endcsname{\let\PY@bf=\textbf\def\PY@tc##1{\textcolor[rgb]{0.00,0.00,1.00}{##1}}}
\expandafter\def\csname PY@tok@nn\endcsname{\let\PY@bf=\textbf\def\PY@tc##1{\textcolor[rgb]{0.00,0.00,1.00}{##1}}}
\expandafter\def\csname PY@tok@ne\endcsname{\let\PY@bf=\textbf\def\PY@tc##1{\textcolor[rgb]{0.82,0.25,0.23}{##1}}}
\expandafter\def\csname PY@tok@nv\endcsname{\def\PY@tc##1{\textcolor[rgb]{0.10,0.09,0.49}{##1}}}
\expandafter\def\csname PY@tok@no\endcsname{\def\PY@tc##1{\textcolor[rgb]{0.53,0.00,0.00}{##1}}}
\expandafter\def\csname PY@tok@nl\endcsname{\def\PY@tc##1{\textcolor[rgb]{0.63,0.63,0.00}{##1}}}
\expandafter\def\csname PY@tok@ni\endcsname{\let\PY@bf=\textbf\def\PY@tc##1{\textcolor[rgb]{0.60,0.60,0.60}{##1}}}
\expandafter\def\csname PY@tok@na\endcsname{\def\PY@tc##1{\textcolor[rgb]{0.49,0.56,0.16}{##1}}}
\expandafter\def\csname PY@tok@nt\endcsname{\let\PY@bf=\textbf\def\PY@tc##1{\textcolor[rgb]{0.00,0.50,0.00}{##1}}}
\expandafter\def\csname PY@tok@nd\endcsname{\def\PY@tc##1{\textcolor[rgb]{0.67,0.13,1.00}{##1}}}
\expandafter\def\csname PY@tok@s\endcsname{\def\PY@tc##1{\textcolor[rgb]{0.73,0.13,0.13}{##1}}}
\expandafter\def\csname PY@tok@sd\endcsname{\let\PY@it=\textit\def\PY@tc##1{\textcolor[rgb]{0.73,0.13,0.13}{##1}}}
\expandafter\def\csname PY@tok@si\endcsname{\let\PY@bf=\textbf\def\PY@tc##1{\textcolor[rgb]{0.73,0.40,0.53}{##1}}}
\expandafter\def\csname PY@tok@se\endcsname{\let\PY@bf=\textbf\def\PY@tc##1{\textcolor[rgb]{0.73,0.40,0.13}{##1}}}
\expandafter\def\csname PY@tok@sr\endcsname{\def\PY@tc##1{\textcolor[rgb]{0.73,0.40,0.53}{##1}}}
\expandafter\def\csname PY@tok@ss\endcsname{\def\PY@tc##1{\textcolor[rgb]{0.10,0.09,0.49}{##1}}}
\expandafter\def\csname PY@tok@sx\endcsname{\def\PY@tc##1{\textcolor[rgb]{0.00,0.50,0.00}{##1}}}
\expandafter\def\csname PY@tok@m\endcsname{\def\PY@tc##1{\textcolor[rgb]{0.40,0.40,0.40}{##1}}}
\expandafter\def\csname PY@tok@gh\endcsname{\let\PY@bf=\textbf\def\PY@tc##1{\textcolor[rgb]{0.00,0.00,0.50}{##1}}}
\expandafter\def\csname PY@tok@gu\endcsname{\let\PY@bf=\textbf\def\PY@tc##1{\textcolor[rgb]{0.50,0.00,0.50}{##1}}}
\expandafter\def\csname PY@tok@gd\endcsname{\def\PY@tc##1{\textcolor[rgb]{0.63,0.00,0.00}{##1}}}
\expandafter\def\csname PY@tok@gi\endcsname{\def\PY@tc##1{\textcolor[rgb]{0.00,0.63,0.00}{##1}}}
\expandafter\def\csname PY@tok@gr\endcsname{\def\PY@tc##1{\textcolor[rgb]{1.00,0.00,0.00}{##1}}}
\expandafter\def\csname PY@tok@ge\endcsname{\let\PY@it=\textit}
\expandafter\def\csname PY@tok@gs\endcsname{\let\PY@bf=\textbf}
\expandafter\def\csname PY@tok@gp\endcsname{\let\PY@bf=\textbf\def\PY@tc##1{\textcolor[rgb]{0.00,0.00,0.50}{##1}}}
\expandafter\def\csname PY@tok@go\endcsname{\def\PY@tc##1{\textcolor[rgb]{0.53,0.53,0.53}{##1}}}
\expandafter\def\csname PY@tok@gt\endcsname{\def\PY@tc##1{\textcolor[rgb]{0.00,0.27,0.87}{##1}}}
\expandafter\def\csname PY@tok@err\endcsname{\def\PY@bc##1{\setlength{\fboxsep}{0pt}\fcolorbox[rgb]{1.00,0.00,0.00}{1,1,1}{\strut ##1}}}
\expandafter\def\csname PY@tok@kc\endcsname{\let\PY@bf=\textbf\def\PY@tc##1{\textcolor[rgb]{0.00,0.50,0.00}{##1}}}
\expandafter\def\csname PY@tok@kd\endcsname{\let\PY@bf=\textbf\def\PY@tc##1{\textcolor[rgb]{0.00,0.50,0.00}{##1}}}
\expandafter\def\csname PY@tok@kn\endcsname{\let\PY@bf=\textbf\def\PY@tc##1{\textcolor[rgb]{0.00,0.50,0.00}{##1}}}
\expandafter\def\csname PY@tok@kr\endcsname{\let\PY@bf=\textbf\def\PY@tc##1{\textcolor[rgb]{0.00,0.50,0.00}{##1}}}
\expandafter\def\csname PY@tok@bp\endcsname{\def\PY@tc##1{\textcolor[rgb]{0.00,0.50,0.00}{##1}}}
\expandafter\def\csname PY@tok@fm\endcsname{\def\PY@tc##1{\textcolor[rgb]{0.00,0.00,1.00}{##1}}}
\expandafter\def\csname PY@tok@vc\endcsname{\def\PY@tc##1{\textcolor[rgb]{0.10,0.09,0.49}{##1}}}
\expandafter\def\csname PY@tok@vg\endcsname{\def\PY@tc##1{\textcolor[rgb]{0.10,0.09,0.49}{##1}}}
\expandafter\def\csname PY@tok@vi\endcsname{\def\PY@tc##1{\textcolor[rgb]{0.10,0.09,0.49}{##1}}}
\expandafter\def\csname PY@tok@vm\endcsname{\def\PY@tc##1{\textcolor[rgb]{0.10,0.09,0.49}{##1}}}
\expandafter\def\csname PY@tok@sa\endcsname{\def\PY@tc##1{\textcolor[rgb]{0.73,0.13,0.13}{##1}}}
\expandafter\def\csname PY@tok@sb\endcsname{\def\PY@tc##1{\textcolor[rgb]{0.73,0.13,0.13}{##1}}}
\expandafter\def\csname PY@tok@sc\endcsname{\def\PY@tc##1{\textcolor[rgb]{0.73,0.13,0.13}{##1}}}
\expandafter\def\csname PY@tok@dl\endcsname{\def\PY@tc##1{\textcolor[rgb]{0.73,0.13,0.13}{##1}}}
\expandafter\def\csname PY@tok@s2\endcsname{\def\PY@tc##1{\textcolor[rgb]{0.73,0.13,0.13}{##1}}}
\expandafter\def\csname PY@tok@sh\endcsname{\def\PY@tc##1{\textcolor[rgb]{0.73,0.13,0.13}{##1}}}
\expandafter\def\csname PY@tok@s1\endcsname{\def\PY@tc##1{\textcolor[rgb]{0.73,0.13,0.13}{##1}}}
\expandafter\def\csname PY@tok@mb\endcsname{\def\PY@tc##1{\textcolor[rgb]{0.40,0.40,0.40}{##1}}}
\expandafter\def\csname PY@tok@mf\endcsname{\def\PY@tc##1{\textcolor[rgb]{0.40,0.40,0.40}{##1}}}
\expandafter\def\csname PY@tok@mh\endcsname{\def\PY@tc##1{\textcolor[rgb]{0.40,0.40,0.40}{##1}}}
\expandafter\def\csname PY@tok@mi\endcsname{\def\PY@tc##1{\textcolor[rgb]{0.40,0.40,0.40}{##1}}}
\expandafter\def\csname PY@tok@il\endcsname{\def\PY@tc##1{\textcolor[rgb]{0.40,0.40,0.40}{##1}}}
\expandafter\def\csname PY@tok@mo\endcsname{\def\PY@tc##1{\textcolor[rgb]{0.40,0.40,0.40}{##1}}}
\expandafter\def\csname PY@tok@ch\endcsname{\let\PY@it=\textit\def\PY@tc##1{\textcolor[rgb]{0.25,0.50,0.50}{##1}}}
\expandafter\def\csname PY@tok@cm\endcsname{\let\PY@it=\textit\def\PY@tc##1{\textcolor[rgb]{0.25,0.50,0.50}{##1}}}
\expandafter\def\csname PY@tok@cpf\endcsname{\let\PY@it=\textit\def\PY@tc##1{\textcolor[rgb]{0.25,0.50,0.50}{##1}}}
\expandafter\def\csname PY@tok@c1\endcsname{\let\PY@it=\textit\def\PY@tc##1{\textcolor[rgb]{0.25,0.50,0.50}{##1}}}
\expandafter\def\csname PY@tok@cs\endcsname{\let\PY@it=\textit\def\PY@tc##1{\textcolor[rgb]{0.25,0.50,0.50}{##1}}}

\def\PYZbs{\char`\\}
\def\PYZus{\char`\_}
\def\PYZob{\char`\{}
\def\PYZcb{\char`\}}
\def\PYZca{\char`\^}
\def\PYZam{\char`\&}
\def\PYZlt{\char`\<}
\def\PYZgt{\char`\>}
\def\PYZsh{\char`\#}
\def\PYZpc{\char`\%}
\def\PYZdl{\char`\$}
\def\PYZhy{\char`\-}
\def\PYZsq{\char`\'}
\def\PYZdq{\char`\"}
\def\PYZti{\char`\~}
% for compatibility with earlier versions
\def\PYZat{@}
\def\PYZlb{[}
\def\PYZrb{]}
\makeatother


    % Exact colors from NB
    \definecolor{incolor}{rgb}{0.0, 0.0, 0.5}
    \definecolor{outcolor}{rgb}{0.545, 0.0, 0.0}



    
    % Prevent overflowing lines due to hard-to-break entities
    \sloppy 
    % Setup hyperref package
    \hypersetup{
      breaklinks=true,  % so long urls are correctly broken across lines
      colorlinks=true,
      urlcolor=urlcolor,
      linkcolor=linkcolor,
      citecolor=citecolor,
      }
    % Slightly bigger margins than the latex defaults
    
    \geometry{verbose,tmargin=1in,bmargin=1in,lmargin=1in,rmargin=1in}
    
    

    \begin{document}
    
    
    \maketitle
    
    

    
    alunos: Lucas Gomes Flegler e Luiz Antonio Roque Guzzo

    \hypertarget{passo-1-esbouxe7ar-o-gruxe1fico-da-funuxe7uxe3o}{%
\section{Passo 1 -- Esboçar o gráfico da
função:}\label{passo-1-esbouxe7ar-o-gruxe1fico-da-funuxe7uxe3o}}

\begin{equation*}
f(x)   = \frac{senx}{x} 
\end{equation*}

    \begin{Verbatim}[commandchars=\\\{\}]
{\color{incolor}In [{\color{incolor}16}]:} \PY{k+kn}{import} \PY{n+nn}{matplotlib}\PY{n+nn}{.}\PY{n+nn}{pyplot} \PY{k}{as} \PY{n+nn}{plt}
         \PY{k+kn}{import} \PY{n+nn}{numpy} \PY{k}{as} \PY{n+nn}{np}
         \PY{k+kn}{import} \PY{n+nn}{texttable} \PY{k}{as} \PY{n+nn}{tt}
         \PY{k+kn}{from} \PY{n+nn}{scipy}\PY{n+nn}{.}\PY{n+nn}{integrate} \PY{k}{import} \PY{n}{quad}
         \PY{k+kn}{import} \PY{n+nn}{csv}
\end{Verbatim}


    \begin{Verbatim}[commandchars=\\\{\}]
{\color{incolor}In [{\color{incolor}2}]:} \PY{k}{def} \PY{n+nf}{gera\PYZus{}grafico}\PY{p}{(}\PY{p}{)}\PY{p}{:}
            \PY{n}{x} \PY{o}{=} \PY{n}{np}\PY{o}{.}\PY{n}{linspace}\PY{p}{(}\PY{o}{\PYZhy{}}\PY{l+m+mi}{3}\PY{o}{*}\PY{n}{np}\PY{o}{.}\PY{n}{pi}\PY{p}{,} \PY{p}{(}\PY{l+m+mi}{3}\PY{o}{*}\PY{n}{np}\PY{o}{.}\PY{n}{pi}\PY{p}{)}\PY{p}{,} \PY{l+m+mi}{200}\PY{p}{)}
            \PY{n}{y} \PY{o}{=} \PY{n}{np}\PY{o}{.}\PY{n}{sin}\PY{p}{(}\PY{n}{x}\PY{p}{)}
            \PY{n}{y2} \PY{o}{=} \PY{n}{y} \PY{o}{/} \PY{n}{x}
            \PY{n}{plt}\PY{o}{.}\PY{n}{plot}\PY{p}{(}\PY{n}{x}\PY{p}{,} \PY{n}{y}\PY{p}{,} \PY{l+s+s1}{\PYZsq{}}\PY{l+s+s1}{\PYZhy{}\PYZhy{}r}\PY{l+s+s1}{\PYZsq{}}\PY{p}{,} \PY{n}{label}\PY{o}{=}\PY{l+s+s1}{\PYZsq{}}\PY{l+s+s1}{sen(x)}\PY{l+s+s1}{\PYZsq{}}\PY{p}{)}
            \PY{n}{plt}\PY{o}{.}\PY{n}{plot}\PY{p}{(}\PY{n}{x}\PY{p}{,} \PY{n}{y2}\PY{p}{,} \PY{n}{color}\PY{o}{=}\PY{l+s+s1}{\PYZsq{}}\PY{l+s+s1}{green}\PY{l+s+s1}{\PYZsq{}}\PY{p}{,} \PY{n}{label}\PY{o}{=}\PY{l+s+s1}{\PYZsq{}}\PY{l+s+s1}{sen(x)/x}\PY{l+s+s1}{\PYZsq{}}\PY{p}{)}
            \PY{n}{plt}\PY{o}{.}\PY{n}{xlabel}\PY{p}{(}\PY{l+s+sa}{r}\PY{l+s+s1}{\PYZsq{}}\PY{l+s+s1}{Eixo \PYZdl{}x\PYZdl{}}\PY{l+s+s1}{\PYZsq{}}\PY{p}{)}
            \PY{n}{plt}\PY{o}{.}\PY{n}{ylabel}\PY{p}{(}\PY{l+s+s1}{\PYZsq{}}\PY{l+s+s1}{Eixo y}\PY{l+s+s1}{\PYZsq{}}\PY{p}{)}
            \PY{n}{plt}\PY{o}{.}\PY{n}{legend}\PY{p}{(}\PY{p}{)}
            \PY{n}{plt}\PY{o}{.}\PY{n}{show}\PY{p}{(}\PY{p}{)}
\end{Verbatim}


    \begin{Verbatim}[commandchars=\\\{\}]
{\color{incolor}In [{\color{incolor}3}]:} \PY{n}{gera\PYZus{}grafico}\PY{p}{(}\PY{p}{)}
\end{Verbatim}


    \begin{center}
    \adjustimage{max size={0.9\linewidth}{0.9\paperheight}}{output_4_0.png}
    \end{center}
    { \hspace*{\fill} \\}
    
    \hypertarget{passo-2-escrever-como-uma-serie-a-funuxe7uxe3o}{%
\section{Passo 2 -- Escrever como uma serie a
função:}\label{passo-2-escrever-como-uma-serie-a-funuxe7uxe3o}}

\begin{equation*}
f(x)   = \frac{senx}{x}
\end{equation*}

    \hypertarget{considere-a-suxe9rie-maclaurin-para}{%
\subsection{Considere a série MacLaurin
para:}\label{considere-a-suxe9rie-maclaurin-para}}

\begin{equation*}
\sin (x) = \sum_{n=0}^\infty (-1)^n \frac{x^{2n+1}}{(2n+1)!}
\end{equation*}

\hypertarget{dividindo-a-serie-por-x-temos}{%
\subsection{dividindo a serie por ``x'',
temos:}\label{dividindo-a-serie-por-x-temos}}

\begin{equation*}
\frac{senx}{x} = \sum_{n=0}^\infty (-1)^n \frac{1}{x} \frac{x^{2n+1}}{(2n+1)!} = \sum_{n=0}^\infty (-1)^n \frac{x^{2n}}{(2n+1)!}
\end{equation*}

\hypertarget{expandindo-a-serie}{%
\subsection{expandindo a serie:}\label{expandindo-a-serie}}

\begin{equation*}
\sum_{n=0}^\infty (-1)^n \frac{x^{2n}}{(2n+1)!} = 1 - \frac{x^2}{3!} + \frac{x^4}{5!} -\frac{x^6}{7!} + ...
\end{equation*}

    \begin{center}\rule{0.5\linewidth}{\linethickness}\end{center}

    \hypertarget{passo-3-escrever-a-derivada-da-funuxe7uxe3o}{%
\section{Passo 3 -- Escrever a derivada da
função:}\label{passo-3-escrever-a-derivada-da-funuxe7uxe3o}}

\begin{equation*}
f(x)   = \frac{senx}{x}
\end{equation*}

\hypertarget{verificar-os-valores-da-derivada-nos-pontos}{%
\subsection{verificar os valores da derivada nos
pontos}\label{verificar-os-valores-da-derivada-nos-pontos}}

\[\begin{array}{|l||c|}
\hline \
x = 0 & x = \pi/2\\
\hline
\hline \
x = \pi & x =3π/2\\
\hline
\hline \
x = 2π & x =5π/2\\
\hline
\hline \
x = 3π & \\
\hline
\end{array}\]

    \hypertarget{derivada-da-serie}{%
\subsection{derivada da serie:}\label{derivada-da-serie}}

\begin{equation*}
\frac{d}{dx}(\sum_{n=0}^\infty (-1)^n \frac{x^{2n}}{(2n+1)!}) = \sum_{n=0}^\infty \frac{(-1)^n 2n}{(2n+1)!}.x^{2n-1}
\end{equation*}

    \begin{Verbatim}[commandchars=\\\{\}]
{\color{incolor}In [{\color{incolor}10}]:} \PY{k}{def} \PY{n+nf}{fatorial}\PY{p}{(}\PY{n}{x}\PY{p}{)}\PY{p}{:}
           \PY{k}{if} \PY{n}{x} \PY{o}{==} \PY{l+m+mi}{0}\PY{p}{:}
             \PY{k}{return} \PY{l+m+mi}{1}
           \PY{k}{return} \PY{n}{x} \PY{o}{*} \PY{n}{fatorial}\PY{p}{(}\PY{n}{x}\PY{o}{\PYZhy{}}\PY{l+m+mi}{1}\PY{p}{)}
         
         \PY{k}{def} \PY{n+nf}{verifica\PYZus{}derivada}\PY{p}{(}\PY{p}{)}\PY{p}{:}
             \PY{n}{tabela} \PY{o}{=} \PY{n}{tt}\PY{o}{.}\PY{n}{Texttable}\PY{p}{(}\PY{p}{)}
             \PY{n}{tabela}\PY{o}{.}\PY{n}{header}\PY{p}{(}\PY{p}{[}\PY{l+s+s1}{\PYZsq{}}\PY{l+s+s1}{x}\PY{l+s+s1}{\PYZsq{}}\PY{p}{,} \PY{l+s+s1}{\PYZsq{}}\PY{l+s+s1}{valor}\PY{l+s+s1}{\PYZsq{}}\PY{p}{]}\PY{p}{)}
             \PY{n}{valores} \PY{o}{=} \PY{p}{[}\PY{l+s+s1}{\PYZsq{}}\PY{l+s+s1}{0}\PY{l+s+s1}{\PYZsq{}}\PY{p}{,}\PY{l+s+s1}{\PYZsq{}}\PY{l+s+s1}{pi/2}\PY{l+s+s1}{\PYZsq{}}\PY{p}{,}\PY{l+s+s1}{\PYZsq{}}\PY{l+s+s1}{pi}\PY{l+s+s1}{\PYZsq{}}\PY{p}{,}\PY{l+s+s1}{\PYZsq{}}\PY{l+s+s1}{3pi/2}\PY{l+s+s1}{\PYZsq{}}\PY{p}{,}\PY{l+s+s1}{\PYZsq{}}\PY{l+s+s1}{2pi}\PY{l+s+s1}{\PYZsq{}}\PY{p}{,}\PY{l+s+s1}{\PYZsq{}}\PY{l+s+s1}{5pi/2}\PY{l+s+s1}{\PYZsq{}}\PY{p}{,}\PY{l+s+s1}{\PYZsq{}}\PY{l+s+s1}{3pi}\PY{l+s+s1}{\PYZsq{}}\PY{p}{]}
             \PY{n}{pontos} \PY{o}{=} \PY{p}{[}\PY{l+m+mf}{0.0}\PY{p}{,} \PY{p}{(}\PY{n}{np}\PY{o}{.}\PY{n}{pi}\PY{o}{/}\PY{l+m+mi}{2}\PY{p}{)}\PY{p}{,} \PY{p}{(}\PY{n}{np}\PY{o}{.}\PY{n}{pi}\PY{p}{)}\PY{p}{,} \PY{p}{(}\PY{l+m+mi}{3}\PY{o}{*}\PY{n}{np}\PY{o}{.}\PY{n}{pi}\PY{p}{)}\PY{o}{/}\PY{l+m+mi}{2}\PY{p}{,} \PY{p}{(}\PY{l+m+mi}{2}\PY{o}{*}\PY{n}{np}\PY{o}{.}\PY{n}{pi}\PY{p}{)}\PY{p}{,} \PY{p}{(}\PY{l+m+mi}{5}\PY{o}{*}\PY{n}{np}\PY{o}{.}\PY{n}{pi}\PY{p}{)}\PY{o}{/}\PY{l+m+mi}{2}\PY{p}{,} \PY{p}{(}\PY{l+m+mi}{3}\PY{o}{*}\PY{n}{np}\PY{o}{.}\PY{n}{pi}\PY{p}{)}\PY{p}{]}
             \PY{n}{resultado} \PY{o}{=} \PY{l+m+mf}{0.0}
             \PY{n}{i} \PY{o}{=} \PY{l+m+mi}{0}
             \PY{k}{for} \PY{n}{x} \PY{o+ow}{in} \PY{n}{pontos}\PY{p}{:}
                 \PY{n}{v} \PY{o}{=} \PY{n}{valores}\PY{p}{[}\PY{n}{i}\PY{p}{]} \PY{c+c1}{\PYZsh{}variavel criada para deixar ajudar a preencher a tabela}
                 \PY{n}{resultado} \PY{o}{=} \PY{l+m+mf}{0.0}
                 \PY{k}{for} \PY{n}{n} \PY{o+ow}{in} \PY{n+nb}{range}\PY{p}{(}\PY{l+m+mi}{1}\PY{p}{,}\PY{l+m+mi}{20}\PY{p}{)}\PY{p}{:} \PY{c+c1}{\PYZsh{} somatoria de n = 1 até 20}
                     \PY{n}{fat} \PY{o}{=} \PY{n}{fatorial}\PY{p}{(}\PY{p}{(}\PY{p}{(}\PY{l+m+mi}{2}\PY{o}{*}\PY{n}{n}\PY{p}{)}\PY{o}{+}\PY{l+m+mi}{1}\PY{p}{)}\PY{p}{)}
                     \PY{n}{resultado} \PY{o}{=} \PY{n}{resultado} \PY{o}{+} \PY{p}{(}\PY{p}{(}\PY{p}{(}\PY{p}{(}\PY{o}{\PYZhy{}}\PY{l+m+mi}{1}\PY{p}{)} \PY{o}{*}\PY{o}{*} \PY{n}{n}\PY{p}{)} \PY{o}{*} \PY{p}{(}\PY{l+m+mi}{2}\PY{o}{*}\PY{n}{n}\PY{p}{)}\PY{p}{)}\PY{o}{/}\PY{n}{fat}\PY{p}{)}\PY{o}{*}\PY{p}{(}\PY{p}{(}\PY{n}{x}\PY{p}{)} \PY{o}{*}\PY{o}{*} \PY{p}{(}\PY{p}{(}\PY{l+m+mi}{2}\PY{o}{*}\PY{n}{n}\PY{p}{)}\PY{o}{\PYZhy{}}\PY{l+m+mi}{1}\PY{p}{)}\PY{p}{)} 
                 \PY{n}{tabela}\PY{o}{.}\PY{n}{add\PYZus{}row}\PY{p}{(}\PY{p}{[}\PY{n}{v}\PY{p}{,}\PY{n}{resultado}\PY{p}{]}\PY{p}{)} \PY{c+c1}{\PYZsh{}salvando os dados na tabela abaixo}
                 \PY{n}{i}\PY{o}{+}\PY{o}{=}\PY{l+m+mi}{1}
             \PY{n+nb}{print}\PY{p}{(}\PY{n}{tabela}\PY{o}{.}\PY{n}{draw}\PY{p}{(}\PY{p}{)}\PY{p}{)} \PY{c+c1}{\PYZsh{}exibindo resultados}
\end{Verbatim}


    \begin{Verbatim}[commandchars=\\\{\}]
{\color{incolor}In [{\color{incolor}11}]:} \PY{n}{verifica\PYZus{}derivada}\PY{p}{(}\PY{p}{)}
\end{Verbatim}


    \begin{Verbatim}[commandchars=\\\{\}]
+-------+--------+
|   x   | valor  |
+=======+========+
| 0     | 0      |
+-------+--------+
| pi/2  | -0.405 |
+-------+--------+
| pi    | -0.318 |
+-------+--------+
| 3pi/2 | 0.045  |
+-------+--------+
| 2pi   | 0.159  |
+-------+--------+
| 5pi/2 | -0.016 |
+-------+--------+
| 3pi   | -0.106 |
+-------+--------+

    \end{Verbatim}

    \hypertarget{passo-4-gerar-uma-tabela-do-intervalo-03ux3c0-da-integral}{%
\section{Passo 4 -- Gerar uma tabela do intervalo {[}0,3π{]} da
integral}\label{passo-4-gerar-uma-tabela-do-intervalo-03ux3c0-da-integral}}

    \begin{equation*}
\int_{0}^{a} f(x) dx
\end{equation*}

    \begin{Verbatim}[commandchars=\\\{\}]
{\color{incolor}In [{\color{incolor}59}]:} \PY{k}{def} \PY{n+nf}{f}\PY{p}{(}\PY{n}{x}\PY{p}{)}\PY{p}{:}
             \PY{n}{i} \PY{o}{=} \PY{l+m+mi}{1}
             \PY{n}{soma} \PY{o}{=} \PY{l+m+mi}{1}
             \PY{k}{while}\PY{p}{(}\PY{n}{i} \PY{o}{\PYZlt{}} \PY{l+m+mi}{30}\PY{p}{)}\PY{p}{:}
                 \PY{n}{fat} \PY{o}{=} \PY{n}{fatorial}\PY{p}{(}\PY{p}{(}\PY{l+m+mi}{2}\PY{o}{*}\PY{n}{i}\PY{p}{)}\PY{o}{+}\PY{l+m+mi}{1}\PY{p}{)}
                 \PY{n}{soma} \PY{o}{=} \PY{n}{soma} \PY{o}{+} \PY{p}{(}\PY{p}{(}\PY{p}{(}\PY{p}{(}\PY{o}{\PYZhy{}}\PY{l+m+mi}{1}\PY{p}{)} \PY{o}{*}\PY{o}{*} \PY{n}{i}\PY{p}{)}\PY{o}{*}\PY{p}{(}\PY{l+m+mi}{2}\PY{o}{*}\PY{n}{i}\PY{p}{)}\PY{p}{)}\PY{o}{/}\PY{n}{fat}\PY{p}{)} \PY{o}{*} \PY{p}{(}\PY{p}{(}\PY{n}{x}\PY{p}{)}\PY{o}{*}\PY{o}{*}\PY{p}{(}\PY{p}{(}\PY{l+m+mi}{2}\PY{o}{*}\PY{n}{i}\PY{p}{)}\PY{o}{\PYZhy{}}\PY{l+m+mi}{1}\PY{p}{)}\PY{p}{)} \PY{c+c1}{\PYZsh{} concatena uma unica expressao dentro da integral}
                 \PY{n}{i} \PY{o}{+}\PY{o}{=}\PY{l+m+mi}{1}
             \PY{k}{return} \PY{n}{soma} \PY{c+c1}{\PYZsh{} expressao a ser integrada pela biblioteca \PYZdq{}scipy.integrate\PYZdq{}}
         
         \PY{k}{def} \PY{n+nf}{geraTabela}\PY{p}{(}\PY{p}{)}\PY{p}{:}
             \PY{n}{x} \PY{o}{=} \PY{l+m+mi}{0}
             \PY{n}{j} \PY{o}{=} \PY{l+m+mi}{0}
             \PY{n}{i} \PY{o}{=} \PY{l+m+mf}{0.0}
             \PY{n}{vet} \PY{o}{=} \PY{p}{[}\PY{p}{]}
             \PY{n}{matriz} \PY{o}{=} \PY{p}{[}\PY{p}{]}
             \PY{n}{tabela} \PY{o}{=} \PY{n}{tt}\PY{o}{.}\PY{n}{Texttable}\PY{p}{(}\PY{p}{)}
             \PY{k}{while} \PY{p}{(}\PY{n}{i} \PY{o}{\PYZlt{}} \PY{l+m+mi}{4}\PY{p}{)}\PY{p}{:} \PY{c+c1}{\PYZsh{} a variavel i varia de 0.0 ate 3.9 }
                 \PY{k}{while} \PY{p}{(}\PY{n}{j} \PY{o}{\PYZlt{}} \PY{l+m+mi}{10}\PY{p}{)}\PY{p}{:} \PY{c+c1}{\PYZsh{} linha da tabela varia de 1 a 9}
                     \PY{n}{b} \PY{o}{=} \PY{n}{i} \PY{o}{+} \PY{p}{(}\PY{n}{j}\PY{o}{/}\PY{l+m+mf}{100.0}\PY{p}{)} \PY{c+c1}{\PYZsh{} incrementando de 0.01 por 0.01}
                     \PY{n}{resultIntegral} \PY{o}{=} \PY{n}{quad}\PY{p}{(}\PY{n}{f}\PY{p}{,}\PY{l+m+mi}{0}\PY{p}{,}\PY{p}{(}\PY{n}{b}\PY{o}{*}\PY{n}{np}\PY{o}{.}\PY{n}{pi}\PY{p}{)}\PY{p}{)} \PY{c+c1}{\PYZsh{} funcao q calcula a integral onde os parametros sao: (expressao a ser integrada, intervalo de integracao,intervalo de integracao)}
                     \PY{n}{valor} \PY{o}{=} \PY{n}{resultIntegral}\PY{p}{[}\PY{l+m+mi}{0}\PY{p}{]}
                     \PY{n}{vet}\PY{o}{.}\PY{n}{append}\PY{p}{(}\PY{n}{valor}\PY{p}{)} \PY{c+c1}{\PYZsh{}salvando o resultado em um vetor(linha a linha da tabela \PYZdq{}0.1\PYZdq{}, \PYZdq{}0.2\PYZdq{}, \PYZdq{}0.3\PYZdq{}, ...)}
                     \PY{n}{j} \PY{o}{+}\PY{o}{=} \PY{l+m+mi}{1}
                 \PY{n}{matriz}\PY{o}{.}\PY{n}{append}\PY{p}{(}\PY{n}{vet}\PY{p}{)} \PY{c+c1}{\PYZsh{} salvando a linha da tabela em uma matriz.}
                 \PY{n}{tabela}\PY{o}{.}\PY{n}{add\PYZus{}row}\PY{p}{(}\PY{p}{[}\PY{n}{vet}\PY{p}{]}\PY{p}{)} \PY{c+c1}{\PYZsh{}salvando os dados na tabela abaixo}
                 \PY{n}{vet} \PY{o}{=} \PY{p}{[}\PY{p}{]}
                 \PY{n}{i} \PY{o}{+}\PY{o}{=} \PY{l+m+mf}{0.1}
                 \PY{n}{j} \PY{o}{=} \PY{l+m+mi}{0}
             \PY{n}{np}\PY{o}{.}\PY{n}{set\PYZus{}printoptions}\PY{p}{(}\PY{n}{precision}\PY{o}{=}\PY{l+m+mi}{2}\PY{p}{,} \PY{n}{linewidth}\PY{o}{=}\PY{l+m+mi}{200}\PY{p}{)} \PY{c+c1}{\PYZsh{}configurando o modo de visualizacao da matriz, onde o \PYZdq{}precision\PYZdq{} e a quantidade de casas decimais apos a virgula e o \PYZdq{}linewidth\PYZdq{} e quantidade de termos a serem imprimidas em uma linha}
             \PY{n+nb}{print}\PY{p}{(}\PY{n}{np}\PY{o}{.}\PY{n}{matrix}\PY{p}{(}\PY{n}{matriz}\PY{p}{)}\PY{p}{)} \PY{c+c1}{\PYZsh{} usando numpy para imprimir a matriz}
\end{Verbatim}


    \begin{Verbatim}[commandchars=\\\{\}]
{\color{incolor}In [{\color{incolor}60}]:} \PY{n}{geraTabela}\PY{p}{(}\PY{p}{)}
\end{Verbatim}


    \begin{Verbatim}[commandchars=\\\{\}]
[[  0.     0.03   0.06   0.09   0.12   0.15   0.18   0.21   0.24   0.27]
 [  0.3    0.33   0.35   0.38   0.41   0.43   0.46   0.49   0.51   0.54]
 [  0.56   0.59   0.61   0.64   0.66   0.69   0.71   0.73   0.76   0.78]
 [  0.8    0.82   0.85   0.87   0.89   0.91   0.93   0.95   0.97   0.99]
 [  1.01   1.03   1.05   1.07   1.09   1.11   1.13   1.15   1.17   1.19]
 [  1.21   1.23   1.24   1.26   1.28   1.3    1.32   1.34   1.35   1.37]
 [  1.39   1.41   1.43   1.44   1.46   1.48   1.5    1.51   1.53   1.55]
 [  1.57   1.58   1.6    1.62   1.64   1.66   1.67   1.69   1.71   1.73]
 [  1.75   1.77   1.78   1.8    1.82   1.84   1.86   1.88   1.9    1.92]
 [  1.94   1.96   1.98   2.     2.02   2.04   2.06   2.08   2.1    2.12]
 [  2.14   2.16   2.18   2.21   2.23   2.25   2.27   2.3    2.32   2.34]
 [  2.37   2.39   2.41   2.44   2.46   2.49   2.51   2.54   2.56   2.59]
 [  2.61   2.64   2.67   2.69   2.72   2.75   2.77   2.8    2.83   2.86]
 [  2.89   2.91   2.94   2.97   3.     3.03   3.06   3.09   3.12   3.15]
 [  3.18   3.21   3.24   3.28   3.31   3.34   3.37   3.4    3.43   3.47]
 [  3.5    3.53   3.57   3.6    3.63   3.67   3.7    3.73   3.77   3.8 ]
 [  3.84   3.87   3.91   3.94   3.98   4.01   4.05   4.08   4.12   4.15]
 [  4.19   4.23   4.26   4.3    4.33   4.37   4.41   4.44   4.48   4.51]
 [  4.55   4.59   4.62   4.66   4.7    4.73   4.77   4.81   4.84   4.88]
 [  4.92   4.95   4.99   5.03   5.06   5.1    5.14   5.17   5.21   5.25]
 [  5.28   5.32   5.36   5.39   5.43   5.46   5.5    5.54   5.57   5.61]
 [  5.64   5.68   5.72   5.75   5.79   5.82   5.86   5.89   5.93   5.96]
 [  6.     6.03   6.07   6.1    6.13   6.17   6.2    6.24   6.27   6.3 ]
 [  6.34   6.37   6.4    6.44   6.47   6.5    6.54   6.57   6.6    6.63]
 [  6.67   6.7    6.73   6.76   6.79   6.83   6.86   6.89   6.92   6.95]
 [  6.98   7.01   7.04   7.07   7.1    7.13   7.16   7.19   7.22   7.25]
 [  7.28   7.31   7.34   7.37   7.4    7.43   7.46   7.49   7.52   7.55]
 [  7.58   7.61   7.64   7.66   7.69   7.72   7.75   7.78   7.81   7.83]
 [  7.86   7.89   7.92   7.95   7.98   8.     8.03   8.06   8.09   8.12]
 [  8.14   8.17   8.2    8.23   8.26   8.28   8.31   8.34   8.37   8.4 ]
 [  8.42   8.45   8.48   8.51   8.54   8.57   8.59   8.62   8.65   8.68]
 [  8.71   8.74   8.76   8.79   8.82   8.85   8.88   8.91   8.94   8.97]
 [  8.99   9.02   9.05   9.08   9.11   9.14   9.17   9.2    9.23   9.26]
 [  9.29   9.32   9.35   9.38   9.41   9.44   9.47   9.5    9.53   9.56]
 [  9.59   9.62   9.65   9.69   9.72   9.75   9.78   9.81   9.84   9.87]
 [  9.9    9.94   9.97  10.    10.03  10.06  10.1   10.13  10.16  10.19]
 [ 10.23  10.26  10.29  10.32  10.36  10.39  10.42  10.45  10.49  10.52]
 [ 10.55  10.59  10.62  10.65  10.69  10.72  10.75  10.79  10.82  10.86]
 [ 10.89  10.92  10.96  10.99  11.02  11.06  11.09  11.13  11.16  11.19]
 [ 11.23  11.26  11.29  11.33  11.36  11.4   11.43  11.46  11.5   11.53]]

    \end{Verbatim}

    \hypertarget{referuxeancias}{%
\section{Referências}\label{referuxeancias}}

    \begin{itemize}
\tightlist
\item
  \url{https://pt.sharelatex.com/learn/Integrals,_sums_and_limits\#Integrals}
\item
  \url{https://github.com/adam-p/markdown-here/wiki/Markdown-Cheatsheet\#links}
\item
  \url{https://www.symbolab.com/}
\item
  \url{http://www.wolframalpha.com/}
\item
  \url{http://jupyter-notebook.readthedocs.io/en/stable/examples/Notebook/Typesetting\%20Equations.html}
\end{itemize}


    % Add a bibliography block to the postdoc
    
    
    
    \end{document}
