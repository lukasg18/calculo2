
% Default to the notebook output style

    


% Inherit from the specified cell style.




    
\documentclass[11pt]{article}

    
    
    \usepackage[T1]{fontenc}
    % Nicer default font (+ math font) than Computer Modern for most use cases
    \usepackage{mathpazo}

    % Basic figure setup, for now with no caption control since it's done
    % automatically by Pandoc (which extracts ![](path) syntax from Markdown).
    \usepackage{graphicx}
    % We will generate all images so they have a width \maxwidth. This means
    % that they will get their normal width if they fit onto the page, but
    % are scaled down if they would overflow the margins.
    \makeatletter
    \def\maxwidth{\ifdim\Gin@nat@width>\linewidth\linewidth
    \else\Gin@nat@width\fi}
    \makeatother
    \let\Oldincludegraphics\includegraphics
    % Set max figure width to be 80% of text width, for now hardcoded.
    \renewcommand{\includegraphics}[1]{\Oldincludegraphics[width=.8\maxwidth]{#1}}
    % Ensure that by default, figures have no caption (until we provide a
    % proper Figure object with a Caption API and a way to capture that
    % in the conversion process - todo).
    \usepackage{caption}
    \DeclareCaptionLabelFormat{nolabel}{}
    \captionsetup{labelformat=nolabel}

    \usepackage{adjustbox} % Used to constrain images to a maximum size 
    \usepackage{xcolor} % Allow colors to be defined
    \usepackage{enumerate} % Needed for markdown enumerations to work
    \usepackage{geometry} % Used to adjust the document margins
    \usepackage{amsmath} % Equations
    \usepackage{amssymb} % Equations
    \usepackage{textcomp} % defines textquotesingle
    % Hack from http://tex.stackexchange.com/a/47451/13684:
    \AtBeginDocument{%
        \def\PYZsq{\textquotesingle}% Upright quotes in Pygmentized code
    }
    \usepackage{upquote} % Upright quotes for verbatim code
    \usepackage{eurosym} % defines \euro
    \usepackage[mathletters]{ucs} % Extended unicode (utf-8) support
    \usepackage[utf8x]{inputenc} % Allow utf-8 characters in the tex document
    \usepackage{fancyvrb} % verbatim replacement that allows latex
    \usepackage{grffile} % extends the file name processing of package graphics 
                         % to support a larger range 
    % The hyperref package gives us a pdf with properly built
    % internal navigation ('pdf bookmarks' for the table of contents,
    % internal cross-reference links, web links for URLs, etc.)
    \usepackage{hyperref}
    \usepackage{longtable} % longtable support required by pandoc >1.10
    \usepackage{booktabs}  % table support for pandoc > 1.12.2
    \usepackage[inline]{enumitem} % IRkernel/repr support (it uses the enumerate* environment)
    \usepackage[normalem]{ulem} % ulem is needed to support strikethroughs (\sout)
                                % normalem makes italics be italics, not underlines
    

    
    
    % Colors for the hyperref package
    \definecolor{urlcolor}{rgb}{0,.145,.698}
    \definecolor{linkcolor}{rgb}{.71,0.21,0.01}
    \definecolor{citecolor}{rgb}{.12,.54,.11}

    % ANSI colors
    \definecolor{ansi-black}{HTML}{3E424D}
    \definecolor{ansi-black-intense}{HTML}{282C36}
    \definecolor{ansi-red}{HTML}{E75C58}
    \definecolor{ansi-red-intense}{HTML}{B22B31}
    \definecolor{ansi-green}{HTML}{00A250}
    \definecolor{ansi-green-intense}{HTML}{007427}
    \definecolor{ansi-yellow}{HTML}{DDB62B}
    \definecolor{ansi-yellow-intense}{HTML}{B27D12}
    \definecolor{ansi-blue}{HTML}{208FFB}
    \definecolor{ansi-blue-intense}{HTML}{0065CA}
    \definecolor{ansi-magenta}{HTML}{D160C4}
    \definecolor{ansi-magenta-intense}{HTML}{A03196}
    \definecolor{ansi-cyan}{HTML}{60C6C8}
    \definecolor{ansi-cyan-intense}{HTML}{258F8F}
    \definecolor{ansi-white}{HTML}{C5C1B4}
    \definecolor{ansi-white-intense}{HTML}{A1A6B2}

    % commands and environments needed by pandoc snippets
    % extracted from the output of `pandoc -s`
    \providecommand{\tightlist}{%
      \setlength{\itemsep}{0pt}\setlength{\parskip}{0pt}}
    \DefineVerbatimEnvironment{Highlighting}{Verbatim}{commandchars=\\\{\}}
    % Add ',fontsize=\small' for more characters per line
    \newenvironment{Shaded}{}{}
    \newcommand{\KeywordTok}[1]{\textcolor[rgb]{0.00,0.44,0.13}{\textbf{{#1}}}}
    \newcommand{\DataTypeTok}[1]{\textcolor[rgb]{0.56,0.13,0.00}{{#1}}}
    \newcommand{\DecValTok}[1]{\textcolor[rgb]{0.25,0.63,0.44}{{#1}}}
    \newcommand{\BaseNTok}[1]{\textcolor[rgb]{0.25,0.63,0.44}{{#1}}}
    \newcommand{\FloatTok}[1]{\textcolor[rgb]{0.25,0.63,0.44}{{#1}}}
    \newcommand{\CharTok}[1]{\textcolor[rgb]{0.25,0.44,0.63}{{#1}}}
    \newcommand{\StringTok}[1]{\textcolor[rgb]{0.25,0.44,0.63}{{#1}}}
    \newcommand{\CommentTok}[1]{\textcolor[rgb]{0.38,0.63,0.69}{\textit{{#1}}}}
    \newcommand{\OtherTok}[1]{\textcolor[rgb]{0.00,0.44,0.13}{{#1}}}
    \newcommand{\AlertTok}[1]{\textcolor[rgb]{1.00,0.00,0.00}{\textbf{{#1}}}}
    \newcommand{\FunctionTok}[1]{\textcolor[rgb]{0.02,0.16,0.49}{{#1}}}
    \newcommand{\RegionMarkerTok}[1]{{#1}}
    \newcommand{\ErrorTok}[1]{\textcolor[rgb]{1.00,0.00,0.00}{\textbf{{#1}}}}
    \newcommand{\NormalTok}[1]{{#1}}
    
    % Additional commands for more recent versions of Pandoc
    \newcommand{\ConstantTok}[1]{\textcolor[rgb]{0.53,0.00,0.00}{{#1}}}
    \newcommand{\SpecialCharTok}[1]{\textcolor[rgb]{0.25,0.44,0.63}{{#1}}}
    \newcommand{\VerbatimStringTok}[1]{\textcolor[rgb]{0.25,0.44,0.63}{{#1}}}
    \newcommand{\SpecialStringTok}[1]{\textcolor[rgb]{0.73,0.40,0.53}{{#1}}}
    \newcommand{\ImportTok}[1]{{#1}}
    \newcommand{\DocumentationTok}[1]{\textcolor[rgb]{0.73,0.13,0.13}{\textit{{#1}}}}
    \newcommand{\AnnotationTok}[1]{\textcolor[rgb]{0.38,0.63,0.69}{\textbf{\textit{{#1}}}}}
    \newcommand{\CommentVarTok}[1]{\textcolor[rgb]{0.38,0.63,0.69}{\textbf{\textit{{#1}}}}}
    \newcommand{\VariableTok}[1]{\textcolor[rgb]{0.10,0.09,0.49}{{#1}}}
    \newcommand{\ControlFlowTok}[1]{\textcolor[rgb]{0.00,0.44,0.13}{\textbf{{#1}}}}
    \newcommand{\OperatorTok}[1]{\textcolor[rgb]{0.40,0.40,0.40}{{#1}}}
    \newcommand{\BuiltInTok}[1]{{#1}}
    \newcommand{\ExtensionTok}[1]{{#1}}
    \newcommand{\PreprocessorTok}[1]{\textcolor[rgb]{0.74,0.48,0.00}{{#1}}}
    \newcommand{\AttributeTok}[1]{\textcolor[rgb]{0.49,0.56,0.16}{{#1}}}
    \newcommand{\InformationTok}[1]{\textcolor[rgb]{0.38,0.63,0.69}{\textbf{\textit{{#1}}}}}
    \newcommand{\WarningTok}[1]{\textcolor[rgb]{0.38,0.63,0.69}{\textbf{\textit{{#1}}}}}
    
    
    % Define a nice break command that doesn't care if a line doesn't already
    % exist.
    \def\br{\hspace*{\fill} \\* }
    % Math Jax compatability definitions
    \def\gt{>}
    \def\lt{<}
    % Document parameters
    \title{Trabalho de Calculo II}
    
    
    

    % Pygments definitions
    
\makeatletter
\def\PY@reset{\let\PY@it=\relax \let\PY@bf=\relax%
    \let\PY@ul=\relax \let\PY@tc=\relax%
    \let\PY@bc=\relax \let\PY@ff=\relax}
\def\PY@tok#1{\csname PY@tok@#1\endcsname}
\def\PY@toks#1+{\ifx\relax#1\empty\else%
    \PY@tok{#1}\expandafter\PY@toks\fi}
\def\PY@do#1{\PY@bc{\PY@tc{\PY@ul{%
    \PY@it{\PY@bf{\PY@ff{#1}}}}}}}
\def\PY#1#2{\PY@reset\PY@toks#1+\relax+\PY@do{#2}}

\expandafter\def\csname PY@tok@w\endcsname{\def\PY@tc##1{\textcolor[rgb]{0.73,0.73,0.73}{##1}}}
\expandafter\def\csname PY@tok@c\endcsname{\let\PY@it=\textit\def\PY@tc##1{\textcolor[rgb]{0.25,0.50,0.50}{##1}}}
\expandafter\def\csname PY@tok@cp\endcsname{\def\PY@tc##1{\textcolor[rgb]{0.74,0.48,0.00}{##1}}}
\expandafter\def\csname PY@tok@k\endcsname{\let\PY@bf=\textbf\def\PY@tc##1{\textcolor[rgb]{0.00,0.50,0.00}{##1}}}
\expandafter\def\csname PY@tok@kp\endcsname{\def\PY@tc##1{\textcolor[rgb]{0.00,0.50,0.00}{##1}}}
\expandafter\def\csname PY@tok@kt\endcsname{\def\PY@tc##1{\textcolor[rgb]{0.69,0.00,0.25}{##1}}}
\expandafter\def\csname PY@tok@o\endcsname{\def\PY@tc##1{\textcolor[rgb]{0.40,0.40,0.40}{##1}}}
\expandafter\def\csname PY@tok@ow\endcsname{\let\PY@bf=\textbf\def\PY@tc##1{\textcolor[rgb]{0.67,0.13,1.00}{##1}}}
\expandafter\def\csname PY@tok@nb\endcsname{\def\PY@tc##1{\textcolor[rgb]{0.00,0.50,0.00}{##1}}}
\expandafter\def\csname PY@tok@nf\endcsname{\def\PY@tc##1{\textcolor[rgb]{0.00,0.00,1.00}{##1}}}
\expandafter\def\csname PY@tok@nc\endcsname{\let\PY@bf=\textbf\def\PY@tc##1{\textcolor[rgb]{0.00,0.00,1.00}{##1}}}
\expandafter\def\csname PY@tok@nn\endcsname{\let\PY@bf=\textbf\def\PY@tc##1{\textcolor[rgb]{0.00,0.00,1.00}{##1}}}
\expandafter\def\csname PY@tok@ne\endcsname{\let\PY@bf=\textbf\def\PY@tc##1{\textcolor[rgb]{0.82,0.25,0.23}{##1}}}
\expandafter\def\csname PY@tok@nv\endcsname{\def\PY@tc##1{\textcolor[rgb]{0.10,0.09,0.49}{##1}}}
\expandafter\def\csname PY@tok@no\endcsname{\def\PY@tc##1{\textcolor[rgb]{0.53,0.00,0.00}{##1}}}
\expandafter\def\csname PY@tok@nl\endcsname{\def\PY@tc##1{\textcolor[rgb]{0.63,0.63,0.00}{##1}}}
\expandafter\def\csname PY@tok@ni\endcsname{\let\PY@bf=\textbf\def\PY@tc##1{\textcolor[rgb]{0.60,0.60,0.60}{##1}}}
\expandafter\def\csname PY@tok@na\endcsname{\def\PY@tc##1{\textcolor[rgb]{0.49,0.56,0.16}{##1}}}
\expandafter\def\csname PY@tok@nt\endcsname{\let\PY@bf=\textbf\def\PY@tc##1{\textcolor[rgb]{0.00,0.50,0.00}{##1}}}
\expandafter\def\csname PY@tok@nd\endcsname{\def\PY@tc##1{\textcolor[rgb]{0.67,0.13,1.00}{##1}}}
\expandafter\def\csname PY@tok@s\endcsname{\def\PY@tc##1{\textcolor[rgb]{0.73,0.13,0.13}{##1}}}
\expandafter\def\csname PY@tok@sd\endcsname{\let\PY@it=\textit\def\PY@tc##1{\textcolor[rgb]{0.73,0.13,0.13}{##1}}}
\expandafter\def\csname PY@tok@si\endcsname{\let\PY@bf=\textbf\def\PY@tc##1{\textcolor[rgb]{0.73,0.40,0.53}{##1}}}
\expandafter\def\csname PY@tok@se\endcsname{\let\PY@bf=\textbf\def\PY@tc##1{\textcolor[rgb]{0.73,0.40,0.13}{##1}}}
\expandafter\def\csname PY@tok@sr\endcsname{\def\PY@tc##1{\textcolor[rgb]{0.73,0.40,0.53}{##1}}}
\expandafter\def\csname PY@tok@ss\endcsname{\def\PY@tc##1{\textcolor[rgb]{0.10,0.09,0.49}{##1}}}
\expandafter\def\csname PY@tok@sx\endcsname{\def\PY@tc##1{\textcolor[rgb]{0.00,0.50,0.00}{##1}}}
\expandafter\def\csname PY@tok@m\endcsname{\def\PY@tc##1{\textcolor[rgb]{0.40,0.40,0.40}{##1}}}
\expandafter\def\csname PY@tok@gh\endcsname{\let\PY@bf=\textbf\def\PY@tc##1{\textcolor[rgb]{0.00,0.00,0.50}{##1}}}
\expandafter\def\csname PY@tok@gu\endcsname{\let\PY@bf=\textbf\def\PY@tc##1{\textcolor[rgb]{0.50,0.00,0.50}{##1}}}
\expandafter\def\csname PY@tok@gd\endcsname{\def\PY@tc##1{\textcolor[rgb]{0.63,0.00,0.00}{##1}}}
\expandafter\def\csname PY@tok@gi\endcsname{\def\PY@tc##1{\textcolor[rgb]{0.00,0.63,0.00}{##1}}}
\expandafter\def\csname PY@tok@gr\endcsname{\def\PY@tc##1{\textcolor[rgb]{1.00,0.00,0.00}{##1}}}
\expandafter\def\csname PY@tok@ge\endcsname{\let\PY@it=\textit}
\expandafter\def\csname PY@tok@gs\endcsname{\let\PY@bf=\textbf}
\expandafter\def\csname PY@tok@gp\endcsname{\let\PY@bf=\textbf\def\PY@tc##1{\textcolor[rgb]{0.00,0.00,0.50}{##1}}}
\expandafter\def\csname PY@tok@go\endcsname{\def\PY@tc##1{\textcolor[rgb]{0.53,0.53,0.53}{##1}}}
\expandafter\def\csname PY@tok@gt\endcsname{\def\PY@tc##1{\textcolor[rgb]{0.00,0.27,0.87}{##1}}}
\expandafter\def\csname PY@tok@err\endcsname{\def\PY@bc##1{\setlength{\fboxsep}{0pt}\fcolorbox[rgb]{1.00,0.00,0.00}{1,1,1}{\strut ##1}}}
\expandafter\def\csname PY@tok@kc\endcsname{\let\PY@bf=\textbf\def\PY@tc##1{\textcolor[rgb]{0.00,0.50,0.00}{##1}}}
\expandafter\def\csname PY@tok@kd\endcsname{\let\PY@bf=\textbf\def\PY@tc##1{\textcolor[rgb]{0.00,0.50,0.00}{##1}}}
\expandafter\def\csname PY@tok@kn\endcsname{\let\PY@bf=\textbf\def\PY@tc##1{\textcolor[rgb]{0.00,0.50,0.00}{##1}}}
\expandafter\def\csname PY@tok@kr\endcsname{\let\PY@bf=\textbf\def\PY@tc##1{\textcolor[rgb]{0.00,0.50,0.00}{##1}}}
\expandafter\def\csname PY@tok@bp\endcsname{\def\PY@tc##1{\textcolor[rgb]{0.00,0.50,0.00}{##1}}}
\expandafter\def\csname PY@tok@fm\endcsname{\def\PY@tc##1{\textcolor[rgb]{0.00,0.00,1.00}{##1}}}
\expandafter\def\csname PY@tok@vc\endcsname{\def\PY@tc##1{\textcolor[rgb]{0.10,0.09,0.49}{##1}}}
\expandafter\def\csname PY@tok@vg\endcsname{\def\PY@tc##1{\textcolor[rgb]{0.10,0.09,0.49}{##1}}}
\expandafter\def\csname PY@tok@vi\endcsname{\def\PY@tc##1{\textcolor[rgb]{0.10,0.09,0.49}{##1}}}
\expandafter\def\csname PY@tok@vm\endcsname{\def\PY@tc##1{\textcolor[rgb]{0.10,0.09,0.49}{##1}}}
\expandafter\def\csname PY@tok@sa\endcsname{\def\PY@tc##1{\textcolor[rgb]{0.73,0.13,0.13}{##1}}}
\expandafter\def\csname PY@tok@sb\endcsname{\def\PY@tc##1{\textcolor[rgb]{0.73,0.13,0.13}{##1}}}
\expandafter\def\csname PY@tok@sc\endcsname{\def\PY@tc##1{\textcolor[rgb]{0.73,0.13,0.13}{##1}}}
\expandafter\def\csname PY@tok@dl\endcsname{\def\PY@tc##1{\textcolor[rgb]{0.73,0.13,0.13}{##1}}}
\expandafter\def\csname PY@tok@s2\endcsname{\def\PY@tc##1{\textcolor[rgb]{0.73,0.13,0.13}{##1}}}
\expandafter\def\csname PY@tok@sh\endcsname{\def\PY@tc##1{\textcolor[rgb]{0.73,0.13,0.13}{##1}}}
\expandafter\def\csname PY@tok@s1\endcsname{\def\PY@tc##1{\textcolor[rgb]{0.73,0.13,0.13}{##1}}}
\expandafter\def\csname PY@tok@mb\endcsname{\def\PY@tc##1{\textcolor[rgb]{0.40,0.40,0.40}{##1}}}
\expandafter\def\csname PY@tok@mf\endcsname{\def\PY@tc##1{\textcolor[rgb]{0.40,0.40,0.40}{##1}}}
\expandafter\def\csname PY@tok@mh\endcsname{\def\PY@tc##1{\textcolor[rgb]{0.40,0.40,0.40}{##1}}}
\expandafter\def\csname PY@tok@mi\endcsname{\def\PY@tc##1{\textcolor[rgb]{0.40,0.40,0.40}{##1}}}
\expandafter\def\csname PY@tok@il\endcsname{\def\PY@tc##1{\textcolor[rgb]{0.40,0.40,0.40}{##1}}}
\expandafter\def\csname PY@tok@mo\endcsname{\def\PY@tc##1{\textcolor[rgb]{0.40,0.40,0.40}{##1}}}
\expandafter\def\csname PY@tok@ch\endcsname{\let\PY@it=\textit\def\PY@tc##1{\textcolor[rgb]{0.25,0.50,0.50}{##1}}}
\expandafter\def\csname PY@tok@cm\endcsname{\let\PY@it=\textit\def\PY@tc##1{\textcolor[rgb]{0.25,0.50,0.50}{##1}}}
\expandafter\def\csname PY@tok@cpf\endcsname{\let\PY@it=\textit\def\PY@tc##1{\textcolor[rgb]{0.25,0.50,0.50}{##1}}}
\expandafter\def\csname PY@tok@c1\endcsname{\let\PY@it=\textit\def\PY@tc##1{\textcolor[rgb]{0.25,0.50,0.50}{##1}}}
\expandafter\def\csname PY@tok@cs\endcsname{\let\PY@it=\textit\def\PY@tc##1{\textcolor[rgb]{0.25,0.50,0.50}{##1}}}

\def\PYZbs{\char`\\}
\def\PYZus{\char`\_}
\def\PYZob{\char`\{}
\def\PYZcb{\char`\}}
\def\PYZca{\char`\^}
\def\PYZam{\char`\&}
\def\PYZlt{\char`\<}
\def\PYZgt{\char`\>}
\def\PYZsh{\char`\#}
\def\PYZpc{\char`\%}
\def\PYZdl{\char`\$}
\def\PYZhy{\char`\-}
\def\PYZsq{\char`\'}
\def\PYZdq{\char`\"}
\def\PYZti{\char`\~}
% for compatibility with earlier versions
\def\PYZat{@}
\def\PYZlb{[}
\def\PYZrb{]}
\makeatother


    % Exact colors from NB
    \definecolor{incolor}{rgb}{0.0, 0.0, 0.5}
    \definecolor{outcolor}{rgb}{0.545, 0.0, 0.0}



    
    % Prevent overflowing lines due to hard-to-break entities
    \sloppy 
    % Setup hyperref package
    \hypersetup{
      breaklinks=true,  % so long urls are correctly broken across lines
      colorlinks=true,
      urlcolor=urlcolor,
      linkcolor=linkcolor,
      citecolor=citecolor,
      }
    % Slightly bigger margins than the latex defaults
    
    \geometry{verbose,tmargin=1in,bmargin=1in,lmargin=1in,rmargin=1in}
    
    

    \begin{document}
    
    
    \maketitle
    
    

    
    Alunos: Lucas Gomes Flegler e Luiz Antonio Roque Guzzo

    \hypertarget{passo-1-esbouxe7ar-o-gruxe1fico-da-funuxe7uxe3o}{%
\section{Passo 1 -- Esboçar o gráfico da
função:}\label{passo-1-esbouxe7ar-o-gruxe1fico-da-funuxe7uxe3o}}

\begin{equation*}
f(x)   = \frac{senx}{x} 
\end{equation*}

    \begin{Verbatim}[commandchars=\\\{\}]
{\color{incolor}In [{\color{incolor}1}]:} \PY{k+kn}{import} \PY{n+nn}{matplotlib}\PY{n+nn}{.}\PY{n+nn}{pyplot} \PY{k}{as} \PY{n+nn}{plt}
        \PY{k+kn}{import} \PY{n+nn}{numpy} \PY{k}{as} \PY{n+nn}{np}
        \PY{k+kn}{import} \PY{n+nn}{pandas} \PY{k}{as} \PY{n+nn}{pd}
        \PY{k+kn}{import} \PY{n+nn}{texttable} \PY{k}{as} \PY{n+nn}{tt}
        \PY{k+kn}{from} \PY{n+nn}{scipy}\PY{n+nn}{.}\PY{n+nn}{integrate} \PY{k}{import} \PY{n}{quad}
        \PY{k+kn}{import} \PY{n+nn}{csv}
\end{Verbatim}


    \begin{Verbatim}[commandchars=\\\{\}]
{\color{incolor}In [{\color{incolor}2}]:} \PY{k}{def} \PY{n+nf}{gera\PYZus{}grafico}\PY{p}{(}\PY{n}{p1}\PY{o}{=}\PY{o}{\PYZhy{}}\PY{l+m+mi}{3}\PY{o}{*}\PY{n}{np}\PY{o}{.}\PY{n}{pi}\PY{p}{,} \PY{n}{p2}\PY{o}{=}\PY{p}{(}\PY{l+m+mi}{3}\PY{o}{*}\PY{n}{np}\PY{o}{.}\PY{n}{pi}\PY{p}{)}\PY{p}{,} \PY{n}{texto}\PY{o}{=}\PY{l+s+s1}{\PYZsq{}}\PY{l+s+s1}{nada}\PY{l+s+s1}{\PYZsq{}}\PY{p}{)}\PY{p}{:}
            \PY{n}{x} \PY{o}{=} \PY{n}{np}\PY{o}{.}\PY{n}{linspace}\PY{p}{(}\PY{n}{p1}\PY{p}{,}\PY{n}{p2}\PY{p}{,}\PY{l+m+mi}{200}\PY{p}{)}
            \PY{n}{y} \PY{o}{=} \PY{n}{np}\PY{o}{.}\PY{n}{sin}\PY{p}{(}\PY{n}{x}\PY{p}{)}
            \PY{n}{y2} \PY{o}{=} \PY{n}{y} \PY{o}{/} \PY{n}{x}
            \PY{n}{plt}\PY{o}{.}\PY{n}{plot}\PY{p}{(}\PY{n}{x}\PY{p}{,} \PY{n}{y}\PY{p}{,} \PY{l+s+s1}{\PYZsq{}}\PY{l+s+s1}{\PYZhy{}\PYZhy{}r}\PY{l+s+s1}{\PYZsq{}}\PY{p}{,} \PY{n}{label}\PY{o}{=}\PY{l+s+s1}{\PYZsq{}}\PY{l+s+s1}{sen(x)}\PY{l+s+s1}{\PYZsq{}}\PY{p}{)}
            \PY{n}{plt}\PY{o}{.}\PY{n}{plot}\PY{p}{(}\PY{n}{x}\PY{p}{,} \PY{n}{y2}\PY{p}{,} \PY{n}{color}\PY{o}{=}\PY{l+s+s1}{\PYZsq{}}\PY{l+s+s1}{green}\PY{l+s+s1}{\PYZsq{}}\PY{p}{,} \PY{n}{label}\PY{o}{=}\PY{l+s+s1}{\PYZsq{}}\PY{l+s+s1}{sen(x)/x}\PY{l+s+s1}{\PYZsq{}}\PY{p}{)}
            \PY{n}{plt}\PY{o}{.}\PY{n}{title}\PY{p}{(}\PY{n}{texto}\PY{p}{)}
            \PY{n}{plt}\PY{o}{.}\PY{n}{xlabel}\PY{p}{(}\PY{l+s+sa}{r}\PY{l+s+s1}{\PYZsq{}}\PY{l+s+s1}{Eixo \PYZdl{}x\PYZdl{}}\PY{l+s+s1}{\PYZsq{}}\PY{p}{)}
            \PY{n}{plt}\PY{o}{.}\PY{n}{ylabel}\PY{p}{(}\PY{l+s+s1}{\PYZsq{}}\PY{l+s+s1}{Eixo y}\PY{l+s+s1}{\PYZsq{}}\PY{p}{)}
            \PY{n}{plt}\PY{o}{.}\PY{n}{legend}\PY{p}{(}\PY{p}{)}
            \PY{n}{plt}\PY{o}{.}\PY{n}{show}\PY{p}{(}\PY{p}{)}
\end{Verbatim}


    \begin{Verbatim}[commandchars=\\\{\}]
{\color{incolor}In [{\color{incolor}3}]:} \PY{n}{gera\PYZus{}grafico}\PY{p}{(}\PY{n}{texto}\PY{o}{=}\PY{l+s+s1}{\PYZsq{}}\PY{l+s+s1}{Intervalo[\PYZhy{}3pi, 3pi]}\PY{l+s+s1}{\PYZsq{}}\PY{p}{)}
\end{Verbatim}


    \begin{center}
    \adjustimage{max size={0.9\linewidth}{0.9\paperheight}}{output_4_0.png}
    \end{center}
    { \hspace*{\fill} \\}
    
    \hypertarget{passo-2-escrever-como-uma-serie-a-funuxe7uxe3o}{%
\section{Passo 2 -- Escrever como uma serie a
função:}\label{passo-2-escrever-como-uma-serie-a-funuxe7uxe3o}}

\begin{equation*}
f(x)   = \frac{senx}{x}
\end{equation*}

    \hypertarget{considere-a-suxe9rie-maclaurin-para}{%
\subsection{Considere a série MacLaurin
para:}\label{considere-a-suxe9rie-maclaurin-para}}

\begin{equation*}
\sin (x) = \sum_{n=0}^\infty (-1)^n \frac{x^{2n+1}}{(2n+1)!}
\end{equation*}

\hypertarget{dividindo-a-serie-por-x-temos}{%
\subsection{dividindo a serie por ``x'',
temos:}\label{dividindo-a-serie-por-x-temos}}

\begin{equation*}
\frac{senx}{x} = \sum_{n=0}^\infty (-1)^n \frac{1}{x} \frac{x^{2n+1}}{(2n+1)!} = \sum_{n=0}^\infty (-1)^n \frac{x^{2n}}{(2n+1)!}
\end{equation*}

\hypertarget{expandindo-a-serie}{%
\subsection{expandindo a serie:}\label{expandindo-a-serie}}

\begin{equation*}
\sum_{n=0}^\infty (-1)^n \frac{x^{2n}}{(2n+1)!} = 1 - \frac{x^2}{3!} + \frac{x^4}{5!} -\frac{x^6}{7!} + ...
\end{equation*}

    \begin{center}\rule{0.5\linewidth}{\linethickness}\end{center}

    \hypertarget{passo-3-escrever-a-derivada-da-funuxe7uxe3o}{%
\section{Passo 3 -- Escrever a derivada da
função:}\label{passo-3-escrever-a-derivada-da-funuxe7uxe3o}}

\begin{equation*}
f(x)   = \frac{senx}{x}
\end{equation*}

\hypertarget{verificar-os-valores-da-derivada-nos-pontos}{%
\subsection{verificar os valores da derivada nos
pontos}\label{verificar-os-valores-da-derivada-nos-pontos}}

\[\begin{array}{|l||c|}
\hline \
x = 0 & x = \pi/2\\
\hline
\hline \
x = \pi & x =3\pi/2\\
\hline
\hline \
x = 2\pi & x =5\pi/2\\
\hline
\hline \
x = 3\pi & \\
\hline
\end{array}\]

    \hypertarget{derivada-da-serie}{%
\subsection{derivada da serie:}\label{derivada-da-serie}}

\begin{equation*}
\frac{d}{dx}(\sum_{n=0}^\infty (-1)^n \frac{x^{2n}}{(2n+1)!}) = \sum_{n=0}^\infty \frac{(-1)^n 2n}{(2n+1)!}.x^{2n-1}
\end{equation*}

    \begin{Verbatim}[commandchars=\\\{\}]
{\color{incolor}In [{\color{incolor}4}]:} \PY{k}{def} \PY{n+nf}{fatorial}\PY{p}{(}\PY{n}{x}\PY{p}{)}\PY{p}{:}
          \PY{k}{if} \PY{n}{x} \PY{o}{==} \PY{l+m+mi}{0}\PY{p}{:}
            \PY{k}{return} \PY{l+m+mi}{1}
          \PY{k}{return} \PY{n}{x} \PY{o}{*} \PY{n}{fatorial}\PY{p}{(}\PY{n}{x}\PY{o}{\PYZhy{}}\PY{l+m+mi}{1}\PY{p}{)}
        
        \PY{k}{def} \PY{n+nf}{verifica\PYZus{}derivada}\PY{p}{(}\PY{p}{)}\PY{p}{:}
            \PY{n}{tabela} \PY{o}{=} \PY{n}{tt}\PY{o}{.}\PY{n}{Texttable}\PY{p}{(}\PY{p}{)}
            \PY{n}{tabela}\PY{o}{.}\PY{n}{header}\PY{p}{(}\PY{p}{[}\PY{l+s+s1}{\PYZsq{}}\PY{l+s+s1}{x}\PY{l+s+s1}{\PYZsq{}}\PY{p}{,} \PY{l+s+s1}{\PYZsq{}}\PY{l+s+s1}{valor}\PY{l+s+s1}{\PYZsq{}}\PY{p}{]}\PY{p}{)}
            \PY{n}{valores} \PY{o}{=} \PY{p}{[}\PY{l+s+s1}{\PYZsq{}}\PY{l+s+s1}{0}\PY{l+s+s1}{\PYZsq{}}\PY{p}{,}\PY{l+s+s1}{\PYZsq{}}\PY{l+s+s1}{pi/2}\PY{l+s+s1}{\PYZsq{}}\PY{p}{,}\PY{l+s+s1}{\PYZsq{}}\PY{l+s+s1}{pi}\PY{l+s+s1}{\PYZsq{}}\PY{p}{,}\PY{l+s+s1}{\PYZsq{}}\PY{l+s+s1}{3pi/2}\PY{l+s+s1}{\PYZsq{}}\PY{p}{,}\PY{l+s+s1}{\PYZsq{}}\PY{l+s+s1}{2pi}\PY{l+s+s1}{\PYZsq{}}\PY{p}{,}\PY{l+s+s1}{\PYZsq{}}\PY{l+s+s1}{5pi/2}\PY{l+s+s1}{\PYZsq{}}\PY{p}{,}\PY{l+s+s1}{\PYZsq{}}\PY{l+s+s1}{3pi}\PY{l+s+s1}{\PYZsq{}}\PY{p}{]}
            \PY{n}{pontos} \PY{o}{=} \PY{p}{[}\PY{l+m+mf}{0.0}\PY{p}{,} \PY{p}{(}\PY{n}{np}\PY{o}{.}\PY{n}{pi}\PY{o}{/}\PY{l+m+mi}{2}\PY{p}{)}\PY{p}{,} \PY{p}{(}\PY{n}{np}\PY{o}{.}\PY{n}{pi}\PY{p}{)}\PY{p}{,} \PY{p}{(}\PY{l+m+mi}{3}\PY{o}{*}\PY{n}{np}\PY{o}{.}\PY{n}{pi}\PY{p}{)}\PY{o}{/}\PY{l+m+mi}{2}\PY{p}{,} \PY{p}{(}\PY{l+m+mi}{2}\PY{o}{*}\PY{n}{np}\PY{o}{.}\PY{n}{pi}\PY{p}{)}\PY{p}{,} \PY{p}{(}\PY{l+m+mi}{5}\PY{o}{*}\PY{n}{np}\PY{o}{.}\PY{n}{pi}\PY{p}{)}\PY{o}{/}\PY{l+m+mi}{2}\PY{p}{,} \PY{p}{(}\PY{l+m+mi}{3}\PY{o}{*}\PY{n}{np}\PY{o}{.}\PY{n}{pi}\PY{p}{)}\PY{p}{]}
            \PY{n}{resultado} \PY{o}{=} \PY{l+m+mf}{0.0}
            \PY{n}{i} \PY{o}{=} \PY{l+m+mi}{0}
            \PY{k}{for} \PY{n}{x} \PY{o+ow}{in} \PY{n}{pontos}\PY{p}{:}
                \PY{n}{v} \PY{o}{=} \PY{n}{valores}\PY{p}{[}\PY{n}{i}\PY{p}{]} \PY{c+c1}{\PYZsh{}variavel criada para deixar ajudar a preencher a tabela}
                \PY{n}{resultado} \PY{o}{=} \PY{l+m+mf}{0.0}
                \PY{k}{for} \PY{n}{n} \PY{o+ow}{in} \PY{n+nb}{range}\PY{p}{(}\PY{l+m+mi}{1}\PY{p}{,}\PY{l+m+mi}{20}\PY{p}{)}\PY{p}{:} \PY{c+c1}{\PYZsh{} somatoria de n = 1 até 20}
                    \PY{n}{fat} \PY{o}{=} \PY{n}{fatorial}\PY{p}{(}\PY{p}{(}\PY{p}{(}\PY{l+m+mi}{2}\PY{o}{*}\PY{n}{n}\PY{p}{)}\PY{o}{+}\PY{l+m+mi}{1}\PY{p}{)}\PY{p}{)}
                    \PY{n}{resultado} \PY{o}{=} \PY{n}{resultado} \PY{o}{+} \PY{p}{(}\PY{p}{(}\PY{p}{(}\PY{p}{(}\PY{o}{\PYZhy{}}\PY{l+m+mi}{1}\PY{p}{)} \PY{o}{*}\PY{o}{*} \PY{n}{n}\PY{p}{)} \PY{o}{*} \PY{p}{(}\PY{l+m+mi}{2}\PY{o}{*}\PY{n}{n}\PY{p}{)}\PY{p}{)}\PY{o}{/}\PY{n}{fat}\PY{p}{)}\PY{o}{*}\PY{p}{(}\PY{p}{(}\PY{n}{x}\PY{p}{)} \PY{o}{*}\PY{o}{*} \PY{p}{(}\PY{p}{(}\PY{l+m+mi}{2}\PY{o}{*}\PY{n}{n}\PY{p}{)}\PY{o}{\PYZhy{}}\PY{l+m+mi}{1}\PY{p}{)}\PY{p}{)} 
                \PY{n}{tabela}\PY{o}{.}\PY{n}{add\PYZus{}row}\PY{p}{(}\PY{p}{[}\PY{n}{v}\PY{p}{,}\PY{n}{resultado}\PY{p}{]}\PY{p}{)} \PY{c+c1}{\PYZsh{}salvando os dados na tabela abaixo}
                \PY{n}{i}\PY{o}{+}\PY{o}{=}\PY{l+m+mi}{1}
            \PY{n+nb}{print}\PY{p}{(}\PY{n}{tabela}\PY{o}{.}\PY{n}{draw}\PY{p}{(}\PY{p}{)}\PY{p}{)} \PY{c+c1}{\PYZsh{}exibindo resultados}
\end{Verbatim}


    \begin{Verbatim}[commandchars=\\\{\}]
{\color{incolor}In [{\color{incolor}5}]:} \PY{n}{verifica\PYZus{}derivada}\PY{p}{(}\PY{p}{)}
        \PY{n}{gera\PYZus{}grafico}\PY{p}{(}\PY{l+m+mf}{0.1}\PY{p}{,} \PY{n}{np}\PY{o}{.}\PY{n}{pi}\PY{p}{,} \PY{n}{texto}\PY{o}{=}\PY{l+s+s1}{\PYZsq{}}\PY{l+s+s1}{Intervalo[0.1, pi]}\PY{l+s+s1}{\PYZsq{}}\PY{p}{)}
        \PY{n}{gera\PYZus{}grafico}\PY{p}{(}\PY{n}{np}\PY{o}{.}\PY{n}{pi}\PY{p}{,} \PY{l+m+mi}{2}\PY{o}{*}\PY{n}{np}\PY{o}{.}\PY{n}{pi}\PY{p}{,} \PY{n}{texto}\PY{o}{=}\PY{l+s+s1}{\PYZsq{}}\PY{l+s+s1}{Intervalo[pi, 2pi]}\PY{l+s+s1}{\PYZsq{}}\PY{p}{)}
        \PY{n}{gera\PYZus{}grafico}\PY{p}{(}\PY{l+m+mi}{2}\PY{o}{*}\PY{n}{np}\PY{o}{.}\PY{n}{pi}\PY{p}{,} \PY{l+m+mi}{3}\PY{o}{*}\PY{n}{np}\PY{o}{.}\PY{n}{pi}\PY{p}{,} \PY{n}{texto}\PY{o}{=}\PY{l+s+s1}{\PYZsq{}}\PY{l+s+s1}{Intervalo[2pi, 3pi]}\PY{l+s+s1}{\PYZsq{}}\PY{p}{)}
        \PY{n}{gera\PYZus{}grafico}\PY{p}{(}\PY{l+m+mi}{3}\PY{o}{*}\PY{n}{np}\PY{o}{.}\PY{n}{pi}\PY{p}{,} \PY{l+m+mi}{4}\PY{o}{*}\PY{n}{np}\PY{o}{.}\PY{n}{pi}\PY{p}{,} \PY{n}{texto}\PY{o}{=}\PY{l+s+s1}{\PYZsq{}}\PY{l+s+s1}{Intervalo[3pi, 4pi]}\PY{l+s+s1}{\PYZsq{}}\PY{p}{)}
\end{Verbatim}


    \begin{Verbatim}[commandchars=\\\{\}]
+-------+--------+
|   x   | valor  |
+=======+========+
| 0     | 0      |
+-------+--------+
| pi/2  | -0.405 |
+-------+--------+
| pi    | -0.318 |
+-------+--------+
| 3pi/2 | 0.045  |
+-------+--------+
| 2pi   | 0.159  |
+-------+--------+
| 5pi/2 | -0.016 |
+-------+--------+
| 3pi   | -0.106 |
+-------+--------+

    \end{Verbatim}

    \begin{center}
    \adjustimage{max size={0.9\linewidth}{0.9\paperheight}}{output_11_1.png}
    \end{center}
    { \hspace*{\fill} \\}
    
    \begin{center}
    \adjustimage{max size={0.9\linewidth}{0.9\paperheight}}{output_11_2.png}
    \end{center}
    { \hspace*{\fill} \\}
    
    \begin{center}
    \adjustimage{max size={0.9\linewidth}{0.9\paperheight}}{output_11_3.png}
    \end{center}
    { \hspace*{\fill} \\}
    
    \begin{center}
    \adjustimage{max size={0.9\linewidth}{0.9\paperheight}}{output_11_4.png}
    \end{center}
    { \hspace*{\fill} \\}
    
    \hypertarget{passo-4-gerar-uma-tabela-do-intervalo-03pi-da-integral}{%
\section{Passo 4 -- Gerar uma tabela do intervalo {[}0,3pi{]} da
integral}\label{passo-4-gerar-uma-tabela-do-intervalo-03pi-da-integral}}

    \begin{equation*}
\int_{0}^{a} f(x) dx
\end{equation*}

    \begin{Verbatim}[commandchars=\\\{\}]
{\color{incolor}In [{\color{incolor}6}]:} \PY{k}{def} \PY{n+nf}{f}\PY{p}{(}\PY{n}{x}\PY{p}{)}\PY{p}{:}
            \PY{n}{i} \PY{o}{=} \PY{l+m+mi}{0}
            \PY{n}{soma} \PY{o}{=} \PY{l+m+mi}{0}
            \PY{k}{while}\PY{p}{(}\PY{n}{i} \PY{o}{\PYZlt{}} \PY{l+m+mi}{30}\PY{p}{)}\PY{p}{:}
                \PY{n}{fat} \PY{o}{=} \PY{n}{fatorial}\PY{p}{(}\PY{p}{(}\PY{l+m+mi}{2}\PY{o}{*}\PY{n}{i}\PY{p}{)}\PY{o}{+}\PY{l+m+mi}{1}\PY{p}{)}
                \PY{c+c1}{\PYZsh{} concatena uma unica expressao dentro da integral}
                \PY{n}{soma} \PY{o}{=} \PY{n}{soma} \PY{o}{+} \PY{p}{(}\PY{p}{(}\PY{o}{\PYZhy{}}\PY{l+m+mi}{1}\PY{p}{)}\PY{o}{*}\PY{o}{*}\PY{n}{i}\PY{p}{)} \PY{o}{*} \PY{p}{(}\PY{p}{(}\PY{p}{(}\PY{n}{x}\PY{p}{)}\PY{o}{*}\PY{o}{*}\PY{p}{(}\PY{p}{(}\PY{l+m+mi}{2}\PY{o}{*}\PY{n}{i}\PY{p}{)}\PY{p}{)}\PY{p}{)}\PY{o}{/}\PY{n}{fat}\PY{p}{)} 
                \PY{n}{i} \PY{o}{+}\PY{o}{=}\PY{l+m+mi}{1}
            \PY{k}{return} \PY{n}{soma} \PY{c+c1}{\PYZsh{} expressao a ser integrada pela biblioteca \PYZdq{}scipy.integrate\PYZdq{}}
        
        \PY{k}{def} \PY{n+nf}{geraTabela}\PY{p}{(}\PY{p}{)}\PY{p}{:}
            \PY{n}{x} \PY{o}{=} \PY{l+m+mi}{0}
            \PY{n}{j} \PY{o}{=} \PY{l+m+mi}{0}
            \PY{n}{i} \PY{o}{=} \PY{l+m+mf}{0.0}
            \PY{n}{vet} \PY{o}{=} \PY{p}{[}\PY{p}{]}
            \PY{n}{matriz} \PY{o}{=} \PY{p}{[}\PY{p}{]}
            \PY{n}{tabela} \PY{o}{=} \PY{n}{tt}\PY{o}{.}\PY{n}{Texttable}\PY{p}{(}\PY{p}{)}
            \PY{k}{while} \PY{p}{(}\PY{n}{i} \PY{o}{\PYZlt{}}\PY{o}{=} \PY{l+m+mi}{3}\PY{p}{)}\PY{p}{:} \PY{c+c1}{\PYZsh{} a variavel i varia de 0.0 ate 3.9 }
                \PY{k}{while} \PY{p}{(}\PY{n}{j} \PY{o}{\PYZlt{}}\PY{o}{=} \PY{l+m+mi}{9}\PY{p}{)}\PY{p}{:} \PY{c+c1}{\PYZsh{} linha da tabela varia de 1 a 9}
                    \PY{n}{b} \PY{o}{=} \PY{n}{i} \PY{o}{+} \PY{p}{(}\PY{n}{j}\PY{o}{/}\PY{l+m+mf}{100.0}\PY{p}{)} \PY{c+c1}{\PYZsh{} incrementando de 0.01 por 0.01}
                    \PY{n}{b} \PY{o}{=} \PY{n}{b} \PY{o}{*} \PY{n}{np}\PY{o}{.}\PY{n}{pi}
                    \PY{c+c1}{\PYZsh{} funcao q calcula a integral onde os parametros sao: }
                    \PY{c+c1}{\PYZsh{} (expressao a ser integrada, intervalo de integracao e }
                    \PY{c+c1}{\PYZsh{} intervalo de integracao)}
                    \PY{n}{resultIntegral} \PY{o}{=} \PY{n}{quad}\PY{p}{(}\PY{n}{f}\PY{p}{,}\PY{l+m+mi}{0}\PY{p}{,}\PY{n}{b}\PY{p}{)}
                    \PY{n}{valor} \PY{o}{=} \PY{n}{resultIntegral}\PY{p}{[}\PY{l+m+mi}{0}\PY{p}{]}
                    \PY{c+c1}{\PYZsh{}salvando o resultado em um }
                    \PY{c+c1}{\PYZsh{}vetor(linha a linha da tabela \PYZdq{}0.1\PYZdq{}, \PYZdq{}0.2\PYZdq{}, \PYZdq{}0.3\PYZdq{}, ...)}
                    \PY{n}{vet}\PY{o}{.}\PY{n}{append}\PY{p}{(}\PY{n}{valor}\PY{p}{)} 
                    \PY{n}{j} \PY{o}{+}\PY{o}{=} \PY{l+m+mi}{1}
                \PY{n}{matriz}\PY{o}{.}\PY{n}{append}\PY{p}{(}\PY{n}{vet}\PY{p}{)} \PY{c+c1}{\PYZsh{} salvando a linha da tabela em uma matriz.}
                \PY{n}{tabela}\PY{o}{.}\PY{n}{add\PYZus{}row}\PY{p}{(}\PY{p}{[}\PY{n}{vet}\PY{p}{]}\PY{p}{)} \PY{c+c1}{\PYZsh{}salvando os dados na tabela abaixo}
                \PY{n}{vet} \PY{o}{=} \PY{p}{[}\PY{p}{]}
                \PY{n}{i} \PY{o}{+}\PY{o}{=} \PY{l+m+mf}{0.1}
                \PY{n}{j} \PY{o}{=} \PY{l+m+mi}{0}
                \PY{c+c1}{\PYZsh{}configurando o modo de visualizacao da matriz}
            \PY{n}{np}\PY{o}{.}\PY{n}{set\PYZus{}printoptions}\PY{p}{(}\PY{n}{precision}\PY{o}{=}\PY{l+m+mi}{6}\PY{p}{,} \PY{n}{linewidth}\PY{o}{=}\PY{l+m+mi}{200}\PY{p}{)} 
            \PY{c+c1}{\PYZsh{} print(np.matrix(matriz)) \PYZsh{} usando numpy para imprimir a matriz}
            \PY{n}{def2} \PY{o}{=} \PY{n}{pd}\PY{o}{.}\PY{n}{DataFrame}\PY{p}{(}\PY{n}{np}\PY{o}{.}\PY{n}{matrix}\PY{p}{(}\PY{n}{matriz}\PY{p}{)}\PY{p}{)}
            \PY{n+nb}{print}\PY{p}{(}\PY{n}{def2}\PY{p}{)}
\end{Verbatim}


    \begin{Verbatim}[commandchars=\\\{\}]
{\color{incolor}In [{\color{incolor}7}]:} \PY{n}{geraTabela}\PY{p}{(}\PY{p}{)}
\end{Verbatim}


    \begin{Verbatim}[commandchars=\\\{\}]
           0         1         2         3         4         5         6  \textbackslash{}
0   0.000000  0.031414  0.062818  0.094201  0.125554  0.156864  0.188124   
1   0.312442  0.343291  0.374027  0.404641  0.435124  0.465464  0.495652   
2   0.614700  0.643988  0.673069  0.701933  0.730572  0.758976  0.787137   
3   0.897189  0.924013  0.950547  0.976782  1.002711  1.028327  1.053623   
4   1.151477  1.175077  1.198319  1.221198  1.243709  1.265846  1.287604   
5   1.370762  1.390562  1.409957  1.428946  1.447525  1.465690  1.483440   
6   1.550234  1.565871  1.581082  1.595864  1.610216  1.624139  1.637632   
7   1.687299  1.698642  1.709558  1.720046  1.730109  1.739748  1.748966   
8   1.781661  1.788804  1.795541  1.801874  1.807808  1.813346  1.818492   
9   1.835237  1.838486  1.841370  1.843894  1.846063  1.847882  1.849358   
10  1.851937  1.851781  1.851317  1.850552  1.849492  1.848144  1.846515   
11  1.837323  1.834392  1.831222  1.827821  1.824195  1.820353  1.816302   
12  1.798159  1.793177  1.788031  1.782730  1.777280  1.771691  1.765969   
13  1.741911  1.735642  1.729286  1.722852  1.716345  1.709775  1.703148   
14  1.676217  1.669414  1.662598  1.655775  1.648951  1.642134  1.635330   
15  1.608373  1.601729  1.595136  1.588599  1.582124  1.575717  1.569382   
16  1.544874  1.538979  1.533187  1.527502  1.521928  1.516469  1.511129   
17  1.491035  1.486345  1.481794  1.477386  1.473124  1.469008  1.465043   
18  1.450724  1.447538  1.444513  1.441650  1.438950  1.436414  1.434042   
19  1.426211  1.424667  1.423289  1.422076  1.421028  1.420144  1.419424   
20  1.418152  1.418230  1.418464  1.418851  1.419390  1.420079  1.420916   
21  1.425691  1.427230  1.428901  1.430702  1.432629  1.434680  1.436851   
22  1.446674  1.449397  1.452219  1.455138  1.458149  1.461249  1.464433   
23  1.477940  1.481490  1.485100  1.488767  1.492487  1.496255  1.500069   
24  1.515684  1.519658  1.523652  1.527661  1.531683  1.535712  1.539744   
25  1.555831  1.559822  1.563794  1.567742  1.571663  1.575553  1.579408   
26  1.594416  1.598048  1.601625  1.605144  1.608603  1.611998  1.615327   
27  1.627922  1.630878  1.633752  1.636543  1.639247  1.641863  1.644389   
28  1.653556  1.655606  1.657556  1.659406  1.661154  1.662798  1.664339   
29  1.669451  1.670463  1.671368  1.672166  1.672857  1.673441  1.673918   

           7         8         9  
0   0.219322  0.250447  0.281491  
1   0.525680  0.555536  0.585213  
2   0.815048  0.842698  0.870082  
3   1.078593  1.103229  1.127526  
4   1.308980  1.329967  1.350563  
5   1.500771  1.517682  1.534170  
6   1.650694  1.663326  1.675527  
7   1.757764  1.766144  1.774109  
8   1.823249  1.827623  1.831617  
9   1.850495  1.851300  1.851779  
10  1.844611  1.842440  1.840008  
11  1.812050  1.807603  1.802970  
12  1.760123  1.754159  1.748086  
13  1.696471  1.689752  1.682998  
14  1.628545  1.621787  1.615061  
15  1.563125  1.556951  1.550866  
16  1.505912  1.500822  1.495862  
17  1.461230  1.457571  1.454069  
18  1.431836  1.429795  1.427920  
19  1.418865  1.418468  1.418230  
20  1.421897  1.423022  1.424288  
21  1.439140  1.441542  1.444054  
22  1.467698  1.471041  1.474456  
23  1.503922  1.507812  1.511734  
24  1.543776  1.547804  1.551824  
25  1.583226  1.587002  1.590733  
26  1.618586  1.621773  1.624886  
27  1.646823  1.649163  1.651408  
28  1.665775  1.667107  1.668332  
29  1.674288  1.674551  1.674709  

    \end{Verbatim}

    \hypertarget{referuxeancias}{%
\section{Referências}\label{referuxeancias}}

    \begin{itemize}
\tightlist
\item
  \url{https://pt.sharelatex.com/learn/Integrals,_sums_and_limits\#Integrals}
\item
  \url{https://github.com/adam-p/markdown-here/wiki/Markdown-Cheatsheet\#links}
\item
  \url{https://www.symbolab.com/}
\item
  \url{http://www.wolframalpha.com/}
\item
  \url{http://jupyter-notebook.readthedocs.io/en/stable/examples/Notebook/Typesetting\%20Equations.html}
\end{itemize}


    % Add a bibliography block to the postdoc
    
    
    
    \end{document}
